
\section*{Общая характеристика работы}

\newcommand{\actuality}{\underline{\textbf{\actualityTXT}}}
\newcommand{\progress}{\underline{\textbf{\progressTXT}}}
\newcommand{\aim}{\underline{{\textbf\aimTXT}}}
\newcommand{\tasks}{\underline{\textbf{\tasksTXT}}}
\newcommand{\novelty}{\underline{\textbf{\noveltyTXT}}}
\newcommand{\influence}{\underline{\textbf{\influenceTXT}}}
\newcommand{\methods}{\underline{\textbf{\methodsTXT}}}
\newcommand{\defpositions}{\underline{\textbf{\defpositionsTXT}}}
\newcommand{\reliability}{\underline{\textbf{\reliabilityTXT}}}
\newcommand{\probation}{\underline{\textbf{\probationTXT}}}
\newcommand{\contribution}{\underline{\textbf{\contributionTXT}}}
\newcommand{\publications}{\underline{\textbf{\publicationsTXT}}}

 Круг задач, решаемых на высокопризводительных вычислительных системах, очень обширен. Он включает в себя множество исследовательских, оборонных и промышленных задач, в частности моделирование астрофизических процессов и физических процессов в плазменных установках, моделирование поведения вещества при сверхвысоких температурах и давлениях, разработка новых материалов, расчеты аэродинамики первспективных самолетов и автомобилей, преджсказание погоды и задачи биоинформатики.
 
 Все эти очень различные задачи объединяет одно: они предъявляют очень высокие требования к оборудованию и системному программному обеспечению вычислительной системы (ВС). При этом одни задачи  отличаются повышенными требованиями апежде всего к скорости вычислений, другие к быстродействию сетевого оборудования, третьи  к скорости работы файловой системы. Таким образом, возникает вопрос об о расчете быстродействия ВС, отдельной ее подсистемы, или в комплексе, и об определении степени пригодности данной ВС для решения того или иного типа задач. 
 
 Это означает необходимость создания специализированных программ-тестов для определения быстродействия конкретной ВС. В настоящее время существует большое количество подобных программ: Linpack, HPCG, NAS parallel benchmark, SPEChpc96.
 Все это многообразие показывает что задача создания теста производительности ВС остается актуальной.
 
 
 Желательно, чтобы тесты создавались на базе программ, используемых для решения реальных задач для того, чтобы результатом работы теста были бы не только числовые показатели, но и конкретное подтвержение способности тестируемой ВС решать сложные задачи, актуальные с научной или производственной точки зрения. Так, тест Lapack создан на основе методов решения системы линейных уравнений с помощью LU-факторизации, тест HPCG - на основе метода сопряженных градиентов, NAS parallel benchmark ориентирован на задачи гидро и газодинамики.
 
 
 
 Кроме тестирования скорости выполнения арифметических операций очень важное значение имеет произвоительность системы ввода-вывода. Это проблема, имеющая самостоятельное научное значение. о чем свидетельствует, в частности, доклад  <<Do you know what your I/O is doing>>. В приложении к конкретным задачам математического моделирования время собственно решения задачи на высокопроизводительной ВС иногда оказывается меньше, чем время записи результатов на жесткие диски.
 Таким образом анализ производительность системы ввода-вывода должен быть включен в тест производительности.
 Одним из немногих тестовых пакетов, решающих эту задачу, является Intel Cluster Checker - набор программ, производящий всестороннее тестирование кластерной системы, и каждой отдельной подсистемы, и вопросы совместного функционирования подсистем. 
 
 Для сокращения временных и ресурсных затрат на проведение тестирования необходимо создавать тесты на базе программ, реализующих такие математические методы, которые предъявляют высокие требования сразу к нескольким, в идеале - ко всем подсистемам тестируемой ВС. 
 
 Одним из таких методов является метод частиц в ячейках.Его достоинством является, с одной стороны, использование в качестве инструментов нескольких существенно различных вычислительных методов, что позволяет исходя из тестированияна методе частиц в ячейках достоверно прогнозировать производительность данной ВС на многих других задачах. С другой стороны, достоинством метода частиук как теста является трудность его эффективной реализации на высокопроизводительных ВС, в частности нерегулярный доступ к памяти, недетерминированный объем и частнота межпроцессорных пересылок и очень больший объем выдач - проблемы с которыми е всегда и не в полном объеме удается справляться на уровне прикладного ПО. Фактическая эффективность работы ВС с таким сложным приложением является хорошей характеристикой качества оборудования и системного ПО.
 
 Таким образом, актуальность работы заключается в необходимости создания именно комплексного теста производительнсти вычислительных систем, охватывающего все аспекты, влияющие на быстродействие: процессорные элементы, шину памяти, модули памяти, параллельную файловую систему, жесткие диски, позволяющего определить  реальную скорость счета с учетом того, что данные находятся не в регистрах процессора и должны быть туда загружены таким образом выяснить реальное быстродействие системы на широком спектре задач математического моделирования на высокопроизводительных вычислительных системах. 


{\aim} данной работы является разработка методики и создание реализующей ее программы для комплекспой оценки производительности ВС с учетом функционирования всех подсистем  и выявления скорости решения реальных задач. Таким образом предполагается представить альтернативу традиционной методике оценки производительности ВВС, основанной только лишь на количестве операций в секунду.  

Для~достижения поставленной цели необходимо было решить следующие {\tasks}:
\begin{enumerate}
  \item Исследовать, разработать, вычислить и~т.\:д. и~т.\:п.
  \item Исследовать, разработать, вычислить и~т.\:д. и~т.\:п.
  \item Исследовать, разработать, вычислить и~т.\:д. и~т.\:п.
  \item Исследовать, разработать, вычислить и~т.\:д. и~т.\:п.
\end{enumerate}


{\novelty}
\begin{enumerate}
  \item Впервые \ldots
  \item Впервые \ldots
  \item Было выполнено оригинальное исследование \ldots
\end{enumerate}

{\influence} \ldots

{\methods} \ldots

{\defpositions}
\begin{enumerate}
  \item Первое положение
  \item Второе положение
  \item Третье положение
  \item Четвертое положение
\end{enumerate}
В папке Documents можно ознакомиться в решением совета из Томского ГУ
в~файле \verb+Def_positions.pdf+, где обоснованно даются рекомендации
по~формулировкам защищаемых положений. 

{\reliability} полученных результатов обеспечивается \ldots \ Результаты находятся в соответствии с результатами, полученными другими авторами.


{\probation}
Основные результаты работы докладывались~на:
перечисление основных конференций, симпозиумов и~т.\:п.

{\contribution} Автор принимал активное участие \ldots

%\publications\ Основные результаты по теме диссертации изложены в ХХ печатных изданиях~\cite{Sokolov,Gaidaenko,Lermontov,Management},
%Х из которых изданы в журналах, рекомендованных ВАК~\cite{Sokolov,Gaidaenko}, 
%ХХ --- в тезисах докладов~\cite{Lermontov,Management}.

\ifnumequal{\value{bibliosel}}{0}{% Встроенная реализация с загрузкой файла через движок bibtex8
    \publications\ Основные результаты по теме диссертации изложены в XX печатных изданиях, 
    X из которых изданы в журналах, рекомендованных ВАК, 
    X "--- в тезисах докладов.%
}{% Реализация пакетом biblatex через движок biber
%Сделана отдельная секция, чтобы не отображались в списке цитированных материалов
    \begin{refsection}[vak,papers,conf]% Подсчет и нумерация авторских работ. Засчитываются только те, которые были прописаны внутри \nocite{}.
        %Чтобы сменить порядок разделов в сгрупированном списке литературы необходимо перетасовать следующие три строчки, а также команды в разделе \newcommand*{\insertbiblioauthorgrouped} в файле biblio/biblatex.tex
        \printbibliography[heading=countauthorvak, env=countauthorvak, keyword=biblioauthorvak, section=1]%
        \printbibliography[heading=countauthorconf, env=countauthorconf, keyword=biblioauthorconf, section=1]%
        \printbibliography[heading=countauthornotvak, env=countauthornotvak, keyword=biblioauthornotvak, section=1]%
        \printbibliography[heading=countauthor, env=countauthor, keyword=biblioauthor, section=1]%
        \nocite{%Порядок перечисления в этом блоке определяет порядок вывода в списке публикаций автора
                vakbib1,vakbib2,vak3multi,VychMethProgExa,SibJVM,AutoParSilan,multigridAuto,VychMetPlasma,Adaptive,VestnikNSU3D,%
                confbib1,confbib2,%
                bib1,bib2,SuperFrI%
        }%
        \publications\ Основные результаты по теме диссертации изложены в~\arabic{citeauthor}~печатных изданиях, 
        \arabic{citeauthorvak} из которых изданы в журналах, рекомендованных ВАК, 
        \arabic{citeauthorconf} "--- в~тезисах докладов.
    \end{refsection}
    \begin{refsection}[vak,papers,conf]%Блок, позволяющий отобрать из всех работ автора наиболее значимые, и только их вывести в автореферате, но считать в блоке выше общее число работ
        \printbibliography[heading=countauthorvak, env=countauthorvak, keyword=biblioauthorvak, section=2]%
        \printbibliography[heading=countauthornotvak, env=countauthornotvak, keyword=biblioauthornotvak, section=2]%
        \printbibliography[heading=countauthorconf, env=countauthorconf, keyword=biblioauthorconf, section=2]%
        \printbibliography[heading=countauthor, env=countauthor, keyword=biblioauthor, section=2]%
        \nocite{vakbib2}%vak
        \nocite{bib1}%notvak
        \nocite{confbib1}%conf
    \end{refsection}
}
При использовании пакета \verb!biblatex! для автоматического подсчёта
количества публикаций автора по теме диссертации, необходимо
их~здесь перечислить с использованием команды \verb!\nocite!.
 % Характеристика работы по структуре во введении и в автореферате не отличается (ГОСТ Р 7.0.11, пункты 5.3.1 и 9.2.1), потому её загружаем из одного и того же внешнего файла, предварительно задав форму выделения некоторым параметрам

%Диссертационная работа была выполнена при поддержке грантов ...

\underline{\textbf{Объем и структура работы.}} Диссертация состоит из~введения,
пяти глав, заключения и~приложения. Полный объем диссертации
\textbf{ХХХ}~страниц текста с~\textbf{ХХ}~рисунками и~5~таблицами. Список литературы содержит \textbf{ХХX}~наименование.

\section*{Содержание работы}
Во \underline{\textbf{введении}} 
обоснована актуальность темы исследования и степень
ее разработанности, сформулированы цели и задачи работы, показана ее научная новизна, теоретическая и практическая значимость; представлены положения, выносимые на защиту, а также степень достоверности и апробация результатов.






\underline{\textbf{Первая глава}} посвящена описанию созданной в рамках диссертационной работы реализация метода частиц в ячейках на параллельных ВС с использованием различных типов ускорителей вычислений. 

\underline{\textbf{Вторая глава}} содержит проведенный обзор литературы по вычислениям, проводимых на современных петафлопсных и перспективных экзафлопсных ВС. В частности выявлены вычислительные алгоритмы, которым уделяется особенное внимание применительно к вычислениям на таких ВС.
В обзорную часть данной диссертационной работы вошло 298 статей за период с 2010 по 2016 год в следующих журналах:Future Generation Computer Systems, Procedia Computers Science, Journal of Parallel and Distributed Computing, Parallel Computing, Journal of Computational Physics, Computer Physics Communications и др.  Распределение статей по приложениям показано в таблице \ref{tab_physics}.

\begin{table}[ht]
	\caption{Распределение по приложениям статей относящихся к тематике <<экзафлопсные вычисления>>}
	\begin{center}
		\begin{tabular}{|c|c|}
			\hline
			вычислительная гидродинамика & 27  \\ \hline 
			ядерные технологии & 17       \\ \hline  
			физика плазмы & 11  \\ \hline 
			разработка новых материалов & 10  \\ \hline 
			предсказание погоды & 10 \\ \hline 
			биомедицинские приложения & 9 \\ \hline 
			астрофизика и космология & 6  \\ \hline 
			молекулярная динамика   & 6   \\ \hline 
			мультифизика & 6              \\ \hline 
			геофизика & 6  \\ \hline 
			финансы & 2  \\ \hline 
		\end{tabular}
	\end{center}
	\label{tab_physics}
\end{table}
Комплексный тест производительности высокопроизводительных ВС должен включать в себя алгоритмы, аналогичные используемым в перечисленных приложениях по создаваемой нагрузке на ВС. Можно показать что метод частиц в ячейках соотвествует этому требованию и таким образом на основании проведенных расчетов по методу частиц в ячейках можно с высокой достоверностью предсказывать эффективность работы ВС на приложениях из приведенного списка. Ограничимся тремя наиболее актуальными (т.е. с наибольшим числом статей приложениями из приведенного списка) и пропуская физику плазмы:
\begin{itemize}
	\item \textbf{Вычислительная гидродинамика:} большая часть численных методов использует вычисления на сетке с регулярным доступом к памяти. Это означает, что нагрузка на процессорные элементы и оперативную память аналогична вычислению электромагнитных полей в методе частиц в ячейках
	\item \textbf{Ядерные технологии}: под этим название в обзоре фигурирует моделирование различных процессов, протекающих в ядерных реакторах выполняемое преимущественно на основе метода Монте-Карло, который является близким аналогом метода частиц в ячейках, с той разницей что для его работы требуется меньшее количество синхронизаций между параллельными процессами
	\item \textbf{Разработка новых материалов}: это квантовохимические расчеты структуры и поведения молекул, вычисление супер-многомерных интегралов в различных приближениях. Нагрузка создаваемая данными методами на ВС может достоверно эмулироваться соответвенно этапом расчета движения частиц и этапом вычисления электромагнитного поля в методе частиц в ячейках.
\end{itemize}	    	
Обобщая, можно сказать, что поскольку метод частиц в ячейках содержит участки кода с высокой вычислительной интенсивностью и с низкой, вычислительные процедуры с регулярным доступом к памяти и с нерегулярным, при использовании кэша разных уровней и без такового, а также параллельные коммуникации всех видов, то с использованием результатов тестирования ВС на методе частиц в ячейках можно предсказать эффективность реализации на данной ВС любого из существующих параллельных вычислительных методов и использующего этот метод физического приложения.

В \underline{\textbf{третьей главе}} описана методика измерения характеристик ВС с помощью программы, релизующей метод частиц в ячейках. 

Предложена методика комплексной оценки тестируемой ВС с точки зрения возможности эффективной реализзации математических моделей на основе определения баланса между скоростью счета и скоростью пересылки данных между узлами ВС. Баланс определяется на основе усреднения данных расчетов по методу частиц в ячейках, который используется в качестве оценки снизу по скорости счета и оценки сверху по памяти для большинства существующих математических методов.

Кроме того, на основе проведенных расчетов измерена скорость счета и скорость перемещения данных для нескольких протестированных ВС.   

\textbf{Расчет производительности процессорных элементов}

Для того, чтобы отделить время счета от времени обращения к оперативной памяти было рассмотрено время работы процедуры,
реализующей одномерное преобразование Фурье, которая является частью физической диагностики, используемой в при моделировании динамики плазмы. Измереннное время с учетом известного размера данных и и количества операций в БПФ (\textit{Е.П.Овсянников и др.}), переводится во флопсы. Сравнительная производительность процессорных элементов некоторых из рассмотренных в диссертационной работе ВС выглядит как показано на рисунке  \ref{procs_flops}:

\begin{figure}[htb]
	\begin{center}
		\includegraphics[height=7cm,keepaspectratio]{images/processor_FLOPS.png}
	\end{center}
	\caption{Производительность процессоров Intel Xeon, измеренная в ходе выполнения одномерного преобразования Фурье на некоторых кластерах. Размерность преобразования $N=64$. Измерения выполнены в 2010 г.}
	\label{procs_flops}
\end{figure} 

\textbf{Расчет производительности коммуникационной сети.}
Разработана методика измерения быстродействия коммуникационной сети на основе анализа времени работы MPI-процедур, осуществляющих обмен граничными значениями между отдельными подобластями при решении уравнений Максвелла и при пересылке модельных частиц. В силу того, что при этом используются различные виды коммуникационных функций  - как блокирующие, так и не блокирующие, как парные, так и коллективные, при использовании эйлерово-лагранжевой декомпозиции - это позволяет набрать в течение одного расчета большую базу данных для получения знаний о структуре коммуникационной сети, времени прохождения сообщений в зависимоссти от размера, системных таймаутах и пр. 

\textbf{Формула для комплексной оценки ВС}
приведено обоснование формулы, на основании которой выносится оценка ВС по материалам проведенных тестов. При этом важно отметить, что оценка является не сравнительной - относительно других ВС, а абсолютной - с точки зрения математического моделирования. 

В частности, для того, чтобы параллельная ВС могла быть признана адаптированной к задачам математического моделирования, она должна соответвовать следующим требованиям:
\begin{enumerate}
	\item Относительно высокая производительность коммуникационной сети, позволяющая пересылать все необходимые для расчета данные, не задерживая вычислений
	\item Очень высокая пропускная способность дисковой подсистемы, обеспечивающая сохранение больших объемов данных, полученных в результате счета  	
\end{enumerate}

Важно отметить, что названы относительные показатели, обеспечивающие возможность пересылать и сохранять данные, без ущерба для скорости вычислений. Именно это и означает  комплексную пригодность ВС к решению задач математического моделирования, т.е. результаты счета сохраняются на диск с той же скоростью, с которой пересылаются данные между узлами данной ВС, и более того, эта скорость не намного меньше скорости вычислений.

Для того, чтобы все три упомянутые величины могли быть использованы в одной формуле, необходимо 
\begin{itemize}
	\item привести эти величины к одной размерности (скорость вычислений выражается во флопсах, скорость обмена данными - в гигабайтах в секунду)
	\item представить обобщенные коэффициенты, позволяющие сравнивать объем данных, сохраняемых на диск и объем данных, пересылаемых по коммуникационной сети ВС  
\end{itemize}


Для решения обоих этих задач использованы усредненные данные расчетов по методу частиц в ячейках на различных ВС. Данный метод может быть использован как оценка снизу, т.е. пригодность некоторой ВС для проведения расчетов по методу частиц в ячейках может трактоваться как возможность проведения расчетов по широкому спектру вычислительных методов, причем, как правило, с большей эффективностью.

Итак, коэффициент перевода из флопсов в байты в секунду для расчетов с частицами равен
$k_{f2b} = 500/576$ = 0.86   
и коээфициент для перевода объема данных, сохраняемых на диск к объему данных, пересылаемых по коммуникационной сети, аналогично, для частиц равен (усредненно):
$k_{MPI} = 0.05$ 
Это объясняется тем, что в среднем не более 5\% частиц пересылается между подобластями.
В итоге формула оценки $\xi$ имеет вид:
$$
\xi = \frac{W_{MPI}} {k_{f2b} W_{PIC}}, 
$$
при условии, что $W_{disc} \approx W_{MPI}$,
здесь $W_{disc}$ - скорость работы дисковой подсистемы (байт/сек), $W_{MPI}$
- скорость пересылки данных по сети - (байт/сек) и $W_{PIC}$ - скорость расчета по частицам (во флопсах).	

\underline{\textbf{Четвертая глава}} посвящена  
анализу масштабируемости, параллельной эффективности и ускорения параллельной ВС

Во второй главе предложена методика интегральной оценки тестируемой ВС с помощью измерения масштабируемости с расчетах по методу частиц в ячейках и определения на основе измерений возрастания потока данных в коммуникационной сети ВС.
%				На основе данных о пересылке модельных частиц измерена производительность коммуникационной сети ВС, при этом важно отметить преимущества использованного метода измерений: пересылка данных имеет высокую степень нерегулярности, а также большой объем, что означает проведение тестирования на большой нагрузке, и возможность широкого применения полученных таким образом данных.
%				Также предложена и апробирована методика определения фактически соседних (с точки зрения MPI) узлов ВС.  

\subsubsection{Определение понятий эффективности, масштабируемости и ускорения}
%\textbf{Степаненко, книжка"высокопроизводительные вычисления", учебник Воеводина}

\textbf{Определение понятий эффективности, масштабируемости и ускорения.}
Выписано определение эффективности параллельной реализации программ, сильной и слабой масштабируемости, ускорения при распараллеливании.

В частности, ускорение параллельного алгоритма для $N$ процессоров, определяется как:
$$
S_N = \frac{T_1(n)}{T_N(n)}
$$  
здесь $T_1$ - время исполнения алгоритма на одном процессоре, $T_N$ - на $N$ процессорах, величина $n$ характеризует вычислительную сложность решаемой задачи, для определения ускорения существенно, что задачи, решаемые на 1 и на $N$ процессорах имеют одинаковую сложность.
Далее, эффективность распараллеливания вычисляется как
$$
\eta_N = \frac{1}{N}\frac{T_1(n)}{T_N(n)}
$$  
эта величина иногда называется эффективностью в сильном смысле (англ. strong efficiency). Кроме того, часто используется эффективность в слабом смысле (англ. weak efficiency):
\begin{equation}
\label{weak_eff}
\eta^{weak}_N = \frac{T_1(1)}{T_N(N)}
\end{equation}
здесь, в отличие от определения ускорения, задачи, решаемые на одном и на $N$ процессорах, имеют различную сложность, т.е. при увеличении количества процессоров в $N$ раз сложность задачи такжэе увеличивается в $N$ раз, и таким образом определяемая эффективность будет 100 \% в том случае, если время вычислений не возрастает при увеличении сложности задачи пропорциональном увеличению количества процессоров.   

Приведены различные используемые варианты этих определений, перечислены факторы, влияющие на  значения этих величин для конкретной ВС, а именно скорость обмена данными и структура коммуникационной сети ВВС, алгоритмы реализации MPI-процедур, в особенности коллективных, настройки коммуникационной системы (таймауты, размер системных буферов, и др.).
%\textbf{написать формулы} 
%\textbf{привести формулы из статьи в СибЖВМ}
Для последующего анализа измереннного времени работы метода частиц в ячейках на параллельной ВС, и для прояснения зависимости этого времени от параметров расчета можно привести следующую схематическую формулу для длительности одного временного шага, аналогично статье (Вшивков В.А. и др., СибЖВМ, 2003):

\begin{equation}
\label{PIC-timestep}
T=T_{F} \left ( \frac{N_x N_y N_z }{P_E}\right )+ T_{F,S}\left (N_y N_z\right ) + T_P\left(\frac{N_p}{P_E\times P_L}\right) +T_{P,S}\left (\frac{N_p*T_p}{P_E\times P_L}\right)
\end{equation}

здесь $T_{F}$ - время вычисления электромагнитного поля, $N_x, N_y, N_z$ - размер расчетной сетки соответственно, по координатам $X$ $Y$ и $Z$, $P_E$ - количество процессоров для эйлеровой декомпозиции, т.е. для разделения расчетной области на подобласти, $T_{F,S}$ - время, затрачиваемое на персылку граничных значений полей и токов между подобластями, которое зависит только от $N_y$ и $N_z$ в силу того что на данный момент используется одномерная декомпозиция,  $T_P$ - время расчета движения частиц, $P_L$ - количество процессов (или потоков) для лагранжевой декомпозиции, т.е. для дополнительного разделения частиц подобласти на группы, вычисляемые на отдельных ядрах.

{Анализ масштабируемости как интегральной характеристики ВС.}

Во втором разделе приведены фактически измеренные на различных высокопроизводительных ВС графики масштабируемости и параллельной эффективности и на основе этих данных проведен анлиз коммуникационной сети данных ВС, в частности, для МВС-100К, рис. \ref{eff2}. 

\begin{figure}[h]
	\begin{center}
		\includegraphics[height=5cm,keepaspectratio]{images/eff_weak_JSCC.png}
		\caption{
			Эффективность распараллеливания в слабом смысле, для МВС-100К, МСЦ РАН.
		}
		\label{eff2}
	\end{center} 
\end{figure}

Возможность проведения анализа коммуникационной сети с помщью расчетов по методу частиц в ячейках основана на известной информации о количестве пересылаемых данных и о виртуальной топологии, используемой в программе.

Размер данных, перемещаемых между двумя соседними MPI-процессами равен $144 \times N_y N_z  $ байт. При этом в идеальном случае, когда соседние MPI-процессы находятся на соседних узлах, коммуникации происходят только между соседними узлами, и поток данных в системе в целом не возрастает с ростом количества используемых в расчете узлов.

В частности, в расчете показанном на рис. \ref{eff2} использована эйлерова декомпозиция. Это означает, что используются только парные пересылки MPI, коллективные пересылки не используются, и поток данных через коммуникационную систему ВС в целом возрастать не должен. Если, тем не менее, он возрастает, что видно на рис. \ref{eff2} в виде снижения эффективности распараллеливания, то это может (при отсутствии коллективных операций), означать, что соседние с точки зрения MPI процессы находятся на физически удаленных друг от друга узлах параллельной ВС.

Обозначая $k_{||}$ зависимость коэффициента при времени пересылок граничных условий от количества процессоров в формуле \ref{PIC-timestep}, так что 
\begin{equation}
T_{F,S} = k_{||} (P_E) \frac{N_y N_z}{P_E}
\end{equation} 
и подставляя формулу \ref{PIC-timestep} d \ref{weak_eff}, рассматривая только лишь время расчета и пересылок электромагнитного поля, можно получить 
\begin{equation}
k_{||} (P_E) = \frac{1}{\eta^{weak}(P_E)} - 1
\end{equation}	  
Таким образом величина $k_{||} (P_E)$  - \textbf{степень нелинейности} коммуникационной структуры параллельной ВС. Фактически она представляет собой отклонение от линейной функции для зависимости времени пересылок от количества процессоров. Он показывает предел возрастания потока данных через коммуникационную структуру ВС при увеличении количества процессоров, используемых в расчете. Эта величина характеризует, в какой степени при передаче информации между соседними процессами в MPI используются узлы параллельной ВС, не являющиеся ближайшими соседями. В силу того, что на значение эффективности оказывает влияние не только свойства оборудования, но и особенност реализации MPI, возникает необходимость разделить эти факторы. Это достигается с помощью привязки процессорв к узлам. 

В итоге, такимобразом определенная  величина $k_{||} $ может быть использована как характеристика параллельной ВС, показывающая реально достижимую с помощью данной ВС эффективность и масштабируемость   

\textbf{Измерение продолжительности параллельных коммуникаций и анализ характеристик и топологии коммуникационного оборудования.}
В этом разделе описано решение задачи об определении соседства процессов по реальным узлам. Это исключительно важно для производительности реальных задач, чтобы виртуально близкие (т.е. по номеру MPI-процесса) процессы исполнялись бы на соседних узлах многопроцессорной ВС. Для этого проводится обмен сообщениями между узлами, выделенными 
для исполнения программы по топологии полного графа, и проводится анализ времени прохождения сообщений, рис. \ref{poly_all2all}.  

Следует отметить, что такого этапа, с обменом сообщениями между всеми процессами в рамках метода частиц нет, поэтому, такой анализ проводится предварительно, перед запуском основной части программы.


\begin{figure}[htb]
	\begin{center}
		\includegraphics[height=7cm,keepaspectratio]{images/polytech_all_to_all.png}
	\end{center}
	\caption{Время пересылок для 40 MPI-процессов, расположенных на 10 узлах по 4 процесса, кластер «Политехник», СПбПУ. По осям X и Y отложены номера процессов, цветовая шкала показывает время пересылок в секундах.}
	\label{poly_all2all}
\end{figure} 
Узлы с минимальным временем считаются близкими, т.е. выясняется фактическая топология ВС. Это сопоставляется с известной информацией о размещении процессов по узлам.	Для повышения производительности приложения в дальнейшем целесообразно передвинуть соседние процессы на те узлы, где по факту меньше задержка по коммуникациям.

%	\textbf{картинку из стьатьи все-со-всеми, (с Политеха)}. 
%	1.измерение всех видов MPI-коммуникаций, сравнение одного с другим (Send, Isend, Bsend) - и увязатиь это с алгоритмом
%	2. варьирование размера сообщений и пр. параметров

%материал статьи НГУ ИТ  с более аакуратным анализом

\textbf{Измерение производительности коммуникационной сети на основе данных о пересылке модельных частиц.}
В \textit{четвертом разделе} приведены результаты измерения скорости пересылок модельных частиц в методе частиц в ячейках и проведенный на основе этого сравнительный анализ скорости работы коммуникационной сети ВС.

На нескольких параллельных ВС были измерены времена, затраченные на пересылку модельных частиц между соседними узлами. Из статитики расчетов по методу частиц в ячейках известно, что пересылается, как правило, не более 5\% модельных частиц. В рассматриваемых расчетах размер сетки $512\times 64 \times 64$ узла  при 150 модельных частицах в ячейке, т.е.  на каждом временном шаге в среднем пересылается 15.7 млн. модельных частиц, что при размере одной частицы в 48 байт означает 720 Мб на каждый временной шаг, что при известном времени пересылки позволяет вычислить скорость. Таким образом была экспериментально измерена величина $T_{P,S}$, входящая в формулу \ref{PIC-timestep}   

\begin{figure}[htb]
	\begin{center}
		\includegraphics[height=7cm,keepaspectratio]{images/particle_send_GBsec.png}
	\end{center}
	\caption{Скорость пересылки данных на некоторых кластерах. Количество модельных частиц: 2.5 млн. на каждое процессорное ядро. Измерения выполнены в 2010 г.}
	\label{procs_flops}
\end{figure}

Измеренная таким образом скорость пресылки данных показывает фактический предел этой величины, реально достижимый для вычислительного приложения с использованием имеющегося оборудования и коммуникационного программного обеспечения. Это подтверждается следующими соображениями:
\begin{itemize}
	\item Объем пересылаемых данных мал: в среднем 6 Мб на процесс
	\item Соседние MPI-процессы, как правило, расположены на близких узлах.
\end{itemize}  	 


\underline{\textbf{Пятая глава}} посвящена 
анализу производительности узлов мультиархитектурной ВС.
В этой главе описаны методы, позволяющие определять скорость счета на ускорителях вычислений и скорость перемещения данных между ускорителем вычислений и хост-машиной, а также давать прогнозы о скорости счета нереализованных еще алгоритмов на тестируемой ВС.

Кроме того, предожена методика оценки качества узлов мультиархитектурной ВС на основе графических ( или других) ускорителей, при этом качество понимается как сбалнсированность средней оценочной скорости счета на ускорителе и скорости пермещения данныхмежду ускоритлем и хостом. 

\textbf{Анализ производительности узлов с графическмим ускорителями}
редставлены результаты анализа производительности узлов с графическими ускорителями на основе измерения времени расчета движения модельных частиц, часть из котрых показана в таблице \ref{PerfGPU}.
			
\begin{table}[ht]
\begin{center}
\caption{Характеристики выполнения основных частей реализации метода частиц на современных GPU}
\begin{tabular}{|c|c|c|c|}
\hline
Название GPU                &  Tesla K80   & Tesla P100 \\ \hline
Расчет электрического поля  &  25.3 мкс    &  16.059    \\ \hline
Сдвиг частиц                &  348 мкс     &  338.16    \\ \hline
Скорость копирования        &              &            \\
частиц с хоста на GPU       & 4.92 ГБ/сек. &8.166 ГБ/сек.  \\ \hline
\end{tabular}
\label{PerfGPU}
\end{center}
\end{table}

Основной вопрос данного раздела, как и всей работы - что можно узнать о данной ВС путем запуска программы, реализующей метод частиц в ячейках? В отличие от 	большинства других разделов информация о характеристиках оборудования в данном случае доступна через стандартный интерфейс, соответственно фактически измеренную скорость счета и скорость пересылки данных между хостом и GPU можно сравнивать с номинальными показателями.

Аналогично разделу \ref{calc_PE} определяется производительность GPU во флопсах как для этапу расчета частиц, так и для этапа расчета электромагнитного поля

Важнейшей интегральной характеристикой ВС, оснащенной графическими ускорителями, является возможность их полноценно использовать. Эта возможность
может быть измерена с помощью сопоставления вычисленной скорости счета и и скорости пересылок данных между хостом и GPU с использованием описанного 
в разделе \ref{complex_estimate} переводного множителя $k_{f2b}$. Этот множитель отражает принципиальную возможность переслать необходимые данные с GPU на хост и далее по коммуникационной сети ВС на соседние узлы раньше, чем они понадобятся для счета на соседнем узле, и таким образом счет может продолжать без задержек, вызванных комммуникациями.

Вместе с тем вопрос, который наиболее часто задают специалисты по математическому моделированию применительно к мультиархитектурной ВС, оснащенной графическими ускорителями - это возможность \textit{эффективной} реализации конкретного вычислительного алгоритма на даннной мультиархитектурной ВС.
Для ответа на данный вопрос предлагается интерполяционная формула:
\begin{equation}
v_{pre} = v_{PIC} k + (1-k) v_{B,E}
\end{equation} 
здесь $ v_{pre}$ - оценка скорости вычислений на GPU для рассматриваемого алгоритма, $v_{PIC}$ - скорость вычислений с на этапе сдвига модельных частиц, $v_{B,E}$ - на этапе расчета электромагнитного поля, а $k$ - интерполяционный множитель, получаемый из следующих соображений.

Как уже говорилось выше, большинство численных методов используемых в математическом моделировании находятся в промежуточном положении по отношению к используемым в методе частиц в ячейках алгоритму вычисления поля и алгоритму расчета движения частиц по следующим показателям:
\begin{itemize}
	\item вычислительной интенсивности (равномерное распределение вычислительно сложных фрагментов по тексту или отдельные высоконагруженные участки);
	\item характеру доступа к оперативнной памяти (регулярный или нерегулярный);
	\item объему используемых данных (большой или маленький).
\end{itemize}
Ориентировочное распределение вычислительных алгоритмов по рассмотренным показателям и соответствующие значения коэффициента $k$ показаны в таблице 


\begin{table}[ht]
\begin{center}
\caption{Определение интерполяционного коэффициента для некоторых типов вычислительных алгоритмов}
\begin{tabular}{|c|c|c|c|c|}
			%	        &   &  &  & k \\ \hline
\hline
Вычислительный    & Интенсивность &  Доступ к           & Объем  & $k$  \\ 
алгоритм          &               &  оперативной  & данных &  \\
                  &               &  памяти       &        &  \\ \hline
Расчет движения   &  низкая       & нерегулярный        & большой & 1.0 \\ 
модельных частиц  &               &                     &          & \\\hline
Метод Монте-Карло &  низкая       & нерегулярный        & средний & 0.9 \\ \hline
Метод SPH         &  низкая       & нерегулярный        & небольшой & 0.6 \\ \hline	Метод             &  высокая      & нерегулярный        & большой & 0.5  \\
конечных элементов &          &              &         & \\ \hline
Конечно-разностные &  высокая  & регулярный & большой & 0.2 \\ 		
схемы (явные)      &           &            &         &     \\\hline
Конечно-разностные &  высокая  & регулярный & большой & 0.1 \\ 		
схемы (явные)-2    &           &            &         &     \\\hline
Вычисление         &  высокая  & регулярный & большой & 0.0 \\ 		
электромагнитного поля      &           &            &         &     \\\hline
			
			
		\end{tabular} 
		\label{tab-interp-koef}              
	\end{center}
\end{table}


%Отднльно рассматривается вопрос о скорости подкачки данных к мультипроцессоорам GPU, т.е. производительность шины памяти, а также скорость загрузки данных из памяти хоста в память GPU. \textbf{таблица из статьи Булл и рассуждения}, вывод о качестве GPU и особ- \textbf{о качестве соединения} 

\textbf{Анализ производительности узлов с многоядерными процессорами и ускорителями вычислений} 
Во \textit{втором разделе} представлены результаты анализа производительности узлов с многоядерными процессорами разных типов и ускорителями вычислений, построенными по технологии, совместимой с x86 (Intel Xeon Phi различных поколений)

Основные вопросы те же, что и разделе, посвященном графическим ускорителям: возможность полноценного использования вычислительной мощности 
ускорителей типа Intel Phi без задержек на перемещение данных и возможность эффективной реализации вычислительных алгоритмов на параллельной ВС, оснащенной ускорителми такого типа. Можно привести таблицу, аналогичную \ref{tab-interp-koef} с той поправкой, что эффективность многопоточного доступа к памяти на Intel Phi несколько ниже, поэтому значения интерполяционого коэффициента будут меньше для алгоритмов, использующих большой объем памяти.






В \underline{\textbf{заключении}} приведены основные результаты работы, которые заключаются в следующем:
%% Согласно ГОСТ Р 7.0.11-2011:
%% 5.3.3 В заключении диссертации излагают итоги выполненного исследования, рекомендации, перспективы дальнейшей разработки темы.
%% 9.2.3 В заключении автореферата диссертации излагают итоги данного исследования, рекомендации и перспективы дальнейшей разработки темы.
в диссертации предложена и реализована оригинальная методика комплексного тестирования мультиархитектурных параллельных вычислительных систем, основанная на используемой в реальных расчетах программе для математического моделирования, а именно на реализации метода частиц в ячейках.
\begin{enumerate}
	\item Создан программный комплекс, основанный на моделировании динамики плазмы методом частиц в ячейках на высокопроизводительных ВС для всестороннего исследования производительности ВС, позволяющий в рамках одного запуска программы определить конкретную подсистему, наиболее заметно снижающую производительность. 
	
	\item Реализован и протестирован метод исследования коммуникационной структуры высокопроизводительных ВС, позволяющий давать рекомендации по более оптимальному распределению процессов в приложении на узлах высокопроизводительной ВС, а также вычислять экcтраполяцию реально полученной производительности на аналогичные системы с большим количеством узлов и процессоров.
	
	\item Разработана и обоснована методика расчета абсолютной оценки пригодности данной ВС для решения реальных задач, основанной на сбалансированности производительности различных подсистем конкретной ВС, в частности оперативной памяти, коммуникационной сети, дисковой подсистемы, процессоров и ускорителей вычислений и позволяющей сравнивать ВС безотносительно используемых программ и решаемых задач. 
	
	\item Предложен и протестирован метод комплексного анализа производительности узлов мультиархитектурной ВС, оснащенной многоядерными процессорами и графическими ускорителями или ускорителями Intel Xeon Phi, основанный на программе для моделирования динамики плазмы методом частиц в ячейках и позволяющий делать прогнозы эффективности данной мультиархитектурной ВС для решения конкретных задач, более достоверные по сравнению с синтетическими тестами.
\end{enumerate}


\textbf{Рекомендации и перспективы дальнейшей разработки темы.}
Основным направлением совершенствования разработанного теста является автоматическая выработка рекомедаций по оптимизации кода, охватывающих не только метод частиц в ячейкаx?  но и остальные наиболее часто используемые в математическом моделировании методы,  под конкретную протестированную ВС.

Одним из наиболее важных вариантов дальнейшего развития программы-теста, созданного в диссертационной работе является перенос на не охваченные в текущем варианте платформы, такие как Android, процессоры архитектуры Sunway, графические ускорители, использующие технологию OpenCL, в частности, AMD Firestream, а также на программируемые логические интегральные схемы (ПЛИС).

Кроме того, необходимостью является апробация теста на крупных высокопроизводительных ВС мощностью более петафлопса, а также - возможно, после специальной адаптации схемы расчета движения частиц - на ВС векторной архитектуры.




\ifdefmacro{\microtypesetup}{\microtypesetup{protrusion=false}}{} % не рекомендуется применять пакет микротипографики к автоматически генерируемому списку литературы
\ifnumequal{\value{bibliosel}}{0}{% Встроенная реализация с загрузкой файла через движок bibtex8
  \renewcommand{\bibname}{\large \authorbibtitle}
  \nocite{*}
  \insertbiblioauthor           % Подключаем Bib-базы
  %\insertbiblioother   % !!! bibtex не умеет работать с несколькими библиографиями !!!
}{% Реализация пакетом biblatex через движок biber
  \ifnumgreater{\value{usefootcite}}{0}{
%  \nocite{*} % Невидимая цитата всех работ, позволит вывести все работы автора
  \insertbiblioauthorcited      % Вывод процитированных в автореферате работ автора
  }{
  \insertbiblioauthor           % Вывод всех работ автора
%  \insertbiblioauthorgrouped    % Вывод всех работ автора, сгруппированных по источникам
%  \insertbiblioauthorimportant  % Вывод наиболее значимых работ автора (определяется в файле characteristic во второй section)
  \insertbiblioother            % Вывод списка литературы, на которую ссылались в тексте автореферата
  }
}
\ifdefmacro{\microtypesetup}{\microtypesetup{protrusion=true}}{}
