\documentclass{article}

\begin{document}

\cite{Lieb2014246}, или разрешение конфликтов при распределении различных видов данных, с различными вариантами доступа к ним, например, %\cite{Sitaraman2016,Balzuweit201667}.
 В отличие от предыдущего пункта, к теме <<Динамическая балансировка загрузки>> отнесены вопросы оптимизации загрузки процессорных элементов, проводимой во время работы программы, например, исследование воздействия несбалансированности загрузки на производительность методов Монте-Карло \cite{Siegel2013901} или реализация метода частиц в ячейках с динамической балансировкой загрузки на языке parallel C \cite{Verleye201310}, а также вопросы, связанные с балансировкой загрузки не только прикладной программы, но и вычислительной системы в целом, что особенно важно с точки зрения перспективы экзафлопсных вычислений \cite{Dong20121254}.
 с учетом того, что различного сорта ошибки будут происходит на системах, состоящих из миллионов процессоров практически каждую секунду. Кроме того, стоимость проведения таких расчетов, как ожидается, будет очень высокой (сейчас расчет продолжительностью в 10 млн. процессоро-часов с использованием метода частиц в ячейках оценивается в 0.5 млн. евро \cite{Vieira}). К этой теме относятся
 обработка контрольных точек, оптимизация и ускорение их чтения и записи \cite{Nicolae2013698,Casanova20157}, диспетчеризация процессов с учетом наличия поврежденных или вышедших из строя процессорных элементов \cite{Defour20161},
 а также анализ производительности реализации MPI, работающей с учетом наличия в системе сбоев \cite{Hursey201215}.
 К последней теме (<<уменьшение количества коммуникаций>>) в рамках направления <<масштабируемость и увеличение числа ПЭ>> отнесены статьи, посвященные созданию параллельных алгоритмов, стремящихся к минимуму коммуникаций (англ. communication-avoiding \cite{Dongarra2013212}). Важность этой темы очевидна для экзафлопсных вычислений: при наличии миллионов процессорных элементов, обменивающихся сообщениями, количество сообщений оценивается (минимально), как $O(N)$, где $N$ - число процессоров, поэтому очень важна разработка алгоритмов, способных или вовсе обойтись без коммуникаций или снизить их количество до $O(1)$. К таким алгоритмам относятся: метод решёточных уравнений Больцмана \cite{Wittmann2013924,Safi2016170}, а также, разработка модели свободной поверхности океана с минимизацией объема коммуникаций\cite{Newman2016877} или решение уравнений типа уравнения Пуассона алгебраическим многосеточным методом, масштабируемым в том смысле, что время решения пропорционально количеству неизвестных (а не количеству ПЭ) \cite{Notay2015237}.
 Сюда относятся численные методы, разработанные специально для экзафлопсных машин (методика локального сгущения шага при решении уравнений реакции-диффузии \cite{Krause2016164}, решение дифференциальных уравнений в частных производных с пониженным объемом коммуникаций \cite{Norman2015}).
 Другая тема  - адаптация имеющихся численных методов для работы на очень большом количестве ПЭ (например, вариант итерационного метода Якоби с ускорением Андерсона \cite{Pratapa201643} или масштабируемые алгоритмы Монте-Карло для финансового моделирования \cite{Alexandrov20111708}). Здесь необходимо ответить на вопрос: где проходить грань между алгоритмами, специально разработанными для экзафлопсных машин, и алгоритмами, адаптированными для экзафлопса. Казалось бы, разница незначительна, и более того, поскольку то, что разрабатывается специально для экзафлопсных суперЭВМ, также разрабатывается не на ровном месте, то можно сказать, что разницы нет вовсе, что это просто одно и то же. Тем не менее  разница здесь фундаментальна, разница та же, что между свойством и определением. Алгоритмы, разработанные специально для экзафлопсных вычислений - это те алгоритмы, разработка которых изначально проводилась с целью минимизировать или свести к нулю коммуникации и обеспечить высокую энергоэффективность  (с достижением физической корректности результатов). С другой стороны, алгоритмы к экзафлопсным вычислениям адаптированные -  это те алгоритмы, которые разрабатывались для параллельных вычислений на десятках или сотнях ПЭ, и которые случайно оказались достаточно хорошо масштабируемыми для их эксплуатации на пета- и экзафлопсных системах. Фундаментальный характер имеющихся различий заключается еще и в том, 
 Также к направлению <<адаптация вычислительных методов к архитектуре суперЭВМ>> отнесены статьи, посвященные работе с данными большого размера. Это один из важнейших вопросов при подготовке к счету на экзафлопсных суперЭВМ, вызванный очевидной необходимостью сохранить результаты счета объемом в сотни петабайт \cite{Abramson20141} и передавать их для обработки. Разделение направления по темам показано в таблице \ref{topic_special}.	
 вычислительных приложений с учетом особенностей вычислительной системы \cite{Poghosyan2015167}, 
 многоуровневый параллелизм \cite{Jacobsen20131,Liu2011261}, реализация параллельного алгоритма с учетом особенностей решаемой физической задачи \cite{Rycerz20131116}, 
 сочетанием различных подходов к параллельной реализации \cite{Chakroun20131563,Jin2011562} и др.      
 В работе \cite{Emad2016} представлен подход, названный <<объединяй и властвуй>>, который заключается в следующем: параллельная программа, реализующая некоторую математическую модель, разбивается на три элемента, а именно вычисления, работу с данными и коммуникации. При этом все три элемента тесно взаимодействуют между собой, но  разрабатываются различным образом и могут быть представлены каждый несколькими компонентами, что обеспечивает адаптацию к структуре суперЭВМ и выбор оптимальных численных методов. 
 В статье \cite{Dosanjh2014} излагаются различные возможные подходы аппаратной точки зрения к построению экзафлопсного суперкомпьютера с  на примере нескольких действующих машин-прототипов с аналогичной архитектурой, но меньшей мощности. Моделирование расчетов на экзафлопсных системах проводится на машинах-прототипах с помощью так называемых мини-приложений. 
 В работе \cite{Subotic2013450} разработана методoлогия упрощения создания и улучшения переносимости приложений для экзафлопсных суперЭВМ, которая называется <<разработка сверху вниз>> (top down methodology). Данная методология основывается на следующем: разработка кода проводится так, чтобы он не был привязан к типу параллелизма или используемому оборудованию. Доработка под реально имеющееся оборудование  проводится на следующей стадии разработки и таким образом код может быть перенесен и запущен на любом типе суперЭВМ.
 Наиболее очевидное применение ускорителей - ускорение расчетов (первая строка в таблице \ref{topic_GPU}), например, вопрос о том, как получить на методе частиц в ячейках заявленную для ускорителя Intel Xeon Phi производительность в 1 Teraflops \cite{Nakashima2015}, или расчеты в области микроволновой томографии для обнаружения рака груди с использованием процессоров IBM Cell \cite{Xu20121106}. Также рассматриваются затруднения, возникающие при работе с памятью при реализации методов Монте-Карло на многоядерных системах \cite{Tramm2015195}, и проблемы эффективной реализации метода частиц в ячейках на GPU \cite{Gong2012588,Kong2011,Safi20151290}.
 В частности, предложен набор программных инструментов для упрощения разработки \cite{Dongarra2015}.
 Кроме того, рассматриваются вопросы переноса программ между различными ускорителями вычислений \cite{Subotic2013} и 
 создание настраиваемых алгоритмов для GPU на примере метода частиц \cite{Decyk2011}. 	 
 Intel Xeon Phi, а также Nvidia Kepler и Nvidia Tesla новыми (или нетрадиционными) типами ускорителей вычислений. К этой теме отнесено сравнение высокоуровневых инструментов реализации численных алгоритмов на ПЛИС \cite{Warne201495} и опыт создания и эксплуатации кластера на процессорах ARM с пониженным энергoпотреблением \cite{Rajovic2014}. Следует особенно отметить, что последняя работа посвящена именно вычислениям на таком кластере и проблемам достижения вычислительной производительности.
 Так или иначе это означает необходимость радикального понижения энергопотребления суперЭВМ. К теме <<энергоэффективность>> в рамках настоящего обзора отнесены следующие работы: исследование производительности методов Монте-Карло с учетом энергопотребления \cite{Atanassov20152719}, методы измерения, моделирования и управления энергопотреблением \cite{Goel20127} и стратегия создания энергоэффективных программ \cite{Trefethen2013}. 
 Также сравнительно небольшое внимание (4.36 \%, таблица \ref{topic_special}) уделяется такому важному вопросу как со-дизайн  и комплексная разработка программ. При этом важно отметить, что лишь две работы из представленного списка (\cite{Emad2016,Subotic2013450}) используют подход со-дизайна полностью, остальные ограничиваются лишь отдельными его элементами, в частности, не производится переработка самого численного метода под архитектуру суперЭВМ (хотя возможность такая рассматривается).
\end{document}