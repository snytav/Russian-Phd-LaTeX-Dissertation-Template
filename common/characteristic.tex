 Круг задач, решаемых на высокопризводительных вычислительных системах, очень обширен. Он включает в себя множество исследовательских, оборонных и промышленных задач, в частности моделирование астрофизических процессов и физических процессов в плазменных установках, моделирование поведения вещества при сверхвысоких температурах и давлениях, разработка новых материалов, расчеты аэродинамики первспективных самолетов и автомобилей, преджсказание погоды и задачи биоинформатики.
 
 Все эти очень различные задачи объединяет одно: они предъявляют очень высокие требования к оборудованию и системному программному обеспечению вычислительной системы (ВС). При этом одни задачи  отличаются повышенными требованиями апежде всего к скорости вычислений, другие к быстродействию сетевого оборудования, третьи  к скорости работы файловой системы. Таким образом, возникает вопрос об о расчете быстродействия ВС, отдельной ее подсистемы, или в комплексе, и об определении степени пригодности данной ВС для решения того или иного типа задач. 
 
 Это означает необходимость создания специализированных программ-тестов для определения быстродействия конкретной ВС. В настоящее время существует большое количество подобных программ: Linpack, HPCG, NAS parallel benchmark, SPEChpc96.
 Все это многообразие показывает что задача создания теста производительности ВС остается актуальной.
 
 
 Желательно, чтобы тесты создавались на базе программ, используемых для решения реальных задач для того, чтобы результатом работы теста были бы не только числовые показатели, но и конкретное подтвержение способности тестируемой ВС решать сложные задачи, актуальные с научной или производственной точки зрения. Так, тест Lapack создан на основе методов решения системы линейных уравнений с помощью LU-факторизации, тест HPCG - на основе метода сопряженных градиентов, NAS parallel benchmark ориентирован на задачи гидро и газодинамики.
 
 
 
 Кроме тестирования скорости выполнения арифметических операций очень важное значение имеет произвоительность системы ввода-вывода. Это проблема, имеющая самостоятельное научное значение. о чем свидетельствует, в частности, доклад  <<Do you know what your I/O is doing>>. В приложении к конкретным задачам математического моделирования время собственно решения задачи на высокопроизводительной ВС иногда оказывается меньше, чем время записи результатов на жесткие диски.
 Таким образом анализ производительность системы ввода-вывода должен быть включен в тест производительности.
 Одним из немногих тестовых пакетов, решающих эту задачу, является Intel Cluster Checker - набор программ, производящий всестороннее тестирование кластерной системы, и каждой отдельной подсистемы, и вопросы совместного функционирования подсистем. 
 
 Для сокращения временных и ресурсных затрат на проведение тестирования необходимо создавать тесты на базе программ, реализующих такие математические методы, которые предъявляют высокие требования сразу к нескольким, в идеале - ко всем подсистемам тестируемой ВС. 
 
 Одним из таких методов является метод частиц в ячейках.Его достоинством является, с одной стороны, использование в качестве инструментов нескольких существенно различных вычислительных методов, что позволяет исходя из тестированияна методе частиц в ячейках достоверно прогнозировать производительность данной ВС на многих других задачах. С другой стороны, достоинством метода частиук как теста является трудность его эффективной реализации на высокопроизводительных ВС, в частности нерегулярный доступ к памяти, недетерминированный объем и частнота межпроцессорных пересылок и очень больший объем выдач - проблемы с которыми е всегда и не в полном объеме удается справляться на уровне прикладного ПО. Фактическая эффективность работы ВС с таким сложным приложением является хорошей характеристикой качества оборудования и системного ПО.
 
 Таким образом, актуальность работы заключается в необходимости создания именно комплексного теста производительнсти вычислительных систем, охватывающего все аспекты, влияющие на быстродействие: процессорные элементы, шину памяти, модули памяти, параллельную файловую систему, жесткие диски, позволяющего определить  реальную скорость счета с учетом того, что данные находятся не в регистрах процессора и должны быть туда загружены таким образом выяснить реальное быстродействие системы на широком спектре задач математического моделирования на высокопроизводительных вычислительных системах. 


{\aim} данной работы является разработка методики и создание реализующей ее программы для комплекспой оценки производительности ВС с учетом функционирования всех подсистем  и выявления скорости решения реальных задач. Таким образом предполагается представить альтернативу традиционной методике оценки производительности ВВС, основанной только лишь на количестве операций в секунду.  

Для~достижения поставленной цели необходимо было решить следующие {\tasks}:
\begin{enumerate}
	\item Создать методику комплексной оценки производительности ВС, основанную на программе для моделирования методом частиц в ячейках.
	\item Разработать и протестировать формулы, обеспечивающие возможность экстраполяции результатов тестирования ускорения и эффективности распараллеливания по части узлов ВС на ВС в целом
	\item Создать методику проведения комплексного анализа производительности узлов мультиархитектурной ВС, оснащенной многоядерными процессорами и ускорителями вычислений на основе метода частиц в ячейках.
	\item Разработать методику обобщения данных, полученных в ходе тестирования на основе метода частиц в ячейках для эффективной эксплутации тестируемой ВС для широкого круга научных и народнохозяйственных задач, а также для для улучшения быстродействия данной ВС. 
\end{enumerate}


{\novelty}
Постановка задачи о проведении тестирования высокопроизводительной ВС на основе метода частиц в ячейках является новой. Впервые создана возможность определения всех факторов, влияющих на производительность ВС на основе одного комплдексного теста. 

Существенная научная новизна определяется тем, что метод частиц в ячейках является ресурсоемким методом, предъявляющим высокие требования к быстродействию не только процессора , но и оперативной памяти, и системы хранения данных. Эти свойства метода частиц приводят к необходимости разработки и применения большого количества сложных методов программной оптимизации, позволяющих достигать приемлемой производительности на реально доступных системах. В то же время высокая требовательность метода частиц к быстродействию ВС позволяет эффективно ценить быстродействие тестируемой ВС с той точки зрения, что если какая-либо из подсистем не оптимизирована и функционирует с низкой производительностью, то итоговая скорость счета по методу частиц будет очень мала. Далее известное разбиение метода частиц в ячейках на этапы различной сложности позволяет определить какая из подситем тестируемой ВС вносит наибольший вклад в снижение производительности. 

Таким образом впервые появляется возможность точной диагностики ВС по результатам работы реализации вычислительного метода, используемой для получения актуальных физических резульатов.   

Преимущества по сравннению с другими вариантами тестирования заключаются в
\begin{itemize}
	\item выявлении производительность ВС на реальном приложении, а не на специально сконструированном тесте - это важно с точки зрения динамики нагрузки на ВС (\textit{объем перемылаемых частиц, например})
	
	\item оценке интегральной производительности ВС, сскладывающейся из производительности процессора, ускорителей вычислений, кэш-памяти разных уровней, оперативной памяти и системы хранения данных
	\item возможности выявления в ходе одного теста конктреной подсистемы ВС, наиболее заметно снижающих скорость счета реальных приложений. 
	
\end{itemize}
Создана оригинальная методика, основанная на анализе статистики сообщений, позволяющая на основе проведения теста на части узлов высокопроизводительной ВС кластерного типа проводить экстраполяцию и оценивать эффективность распараллеливания и масштабируемость для ВС в целом.

Впервые проводится анализ производительности узлов мультиаврхитектурнной ВС, оснащенных графическими ускорителями, ускорителями Intel Xeon Phi, различными типами многоядерных процесоров с точки зрения выявления их производительности, реально достижимой при решения конкретных научно-технических задач.

На основе разделения метода частиц в ячейках на этапы различной сложности, предъявляющих высокие требования к определекнным подсистемам тестируемой ВС создана новая оригинальная методика, позволяющая предсказывать производительность при решении раличных научных и народнохозяйственных задач, в частности задач гидро- и газодинамики, методов Монте-Карло, алгоритмов динамической балансировки загрузки, алгоритмов, основанных на интегральных преобразованиях, задач анализа генетической информации, параллельную эффективность и масштабируемость приложений, использующих парные и коллективные коммуникационные операции. Также разработанный тест может использоваться для получения рекомендации по настройке системы параллельных коммуникаций, например, задания оптимальных значений таймаутов MPI.
{\influence} \ldots

{\methods} \ldots

{\defpositions}
\begin{enumerate}
  \item Первое положение
  \item Второе положение
  \item Третье положение
  \item Четвертое положение
\end{enumerate}
В папке Documents можно ознакомиться в решением совета из Томского ГУ
в~файле \verb+Def_positions.pdf+, где обоснованно даются рекомендации
по~формулировкам защищаемых положений. 

{\reliability} полученных результатов обеспечивается \ldots \ Результаты находятся в соответствии с результатами, полученными другими авторами.


{\probation}
Основные результаты работы докладывались~на:
перечисление основных конференций, симпозиумов и~т.\:п.

{\contribution} Автор принимал активное участие \ldots

%\publications\ Основные результаты по теме диссертации изложены в ХХ печатных изданиях~\cite{Sokolov,Gaidaenko,Lermontov,Management},
%Х из которых изданы в журналах, рекомендованных ВАК~\cite{Sokolov,Gaidaenko}, 
%ХХ --- в тезисах докладов~\cite{Lermontov,Management}.

\ifnumequal{\value{bibliosel}}{0}{% Встроенная реализация с загрузкой файла через движок bibtex8
    \publications\ Основные результаты по теме диссертации изложены в XX печатных изданиях, 
    X из которых изданы в журналах, рекомендованных ВАК, 
    X "--- в тезисах докладов.%
}{% Реализация пакетом biblatex через движок biber
%Сделана отдельная секция, чтобы не отображались в списке цитированных материалов
    \begin{refsection}[vak,papers,conf]% Подсчет и нумерация авторских работ. Засчитываются только те, которые были прописаны внутри \nocite{}.
        %Чтобы сменить порядок разделов в сгрупированном списке литературы необходимо перетасовать следующие три строчки, а также команды в разделе \newcommand*{\insertbiblioauthorgrouped} в файле biblio/biblatex.tex
        \printbibliography[heading=countauthorvak, env=countauthorvak, keyword=biblioauthorvak, section=1]%
        \printbibliography[heading=countauthorconf, env=countauthorconf, keyword=biblioauthorconf, section=1]%
        \printbibliography[heading=countauthornotvak, env=countauthornotvak, keyword=biblioauthornotvak, section=1]%
        \printbibliography[heading=countauthor, env=countauthor, keyword=biblioauthor, section=1]%
        \nocite{%Порядок перечисления в этом блоке определяет порядок вывода в списке публикаций автора
                vakbib1,vakbib2,vak3multi,VychMethProgExa,SibJVM,AutoParSilan,multigridAuto,VychMetPlasma,Adaptive,VestnikNSU3D,%
                confbib1,confbib2,%
                bib1,bib2,SuperFrI%
        }%
        \publications\ Основные результаты по теме диссертации изложены в~\arabic{citeauthor}~печатных изданиях, 
        \arabic{citeauthorvak} из которых изданы в журналах, рекомендованных ВАК, 
        \arabic{citeauthorconf} "--- в~тезисах докладов.
    \end{refsection}
    \begin{refsection}[vak,papers,conf]%Блок, позволяющий отобрать из всех работ автора наиболее значимые, и только их вывести в автореферате, но считать в блоке выше общее число работ
        \printbibliography[heading=countauthorvak, env=countauthorvak, keyword=biblioauthorvak, section=2]%
        \printbibliography[heading=countauthornotvak, env=countauthornotvak, keyword=biblioauthornotvak, section=2]%
        \printbibliography[heading=countauthorconf, env=countauthorconf, keyword=biblioauthorconf, section=2]%
        \printbibliography[heading=countauthor, env=countauthor, keyword=biblioauthor, section=2]%
        \nocite{vakbib2}%vak
        \nocite{bib1}%notvak
        \nocite{confbib1}%conf
    \end{refsection}
}
При использовании пакета \verb!biblatex! для автоматического подсчёта
количества публикаций автора по теме диссертации, необходимо
их~здесь перечислить с использованием команды \verb!\nocite!.
