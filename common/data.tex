%%% Основные сведения %%%
\newcommand{\thesisAuthorLastName}{\todo{Снытников}}
\newcommand{\thesisAuthorOtherNames}{\todo{Алексей Владимирович}}
\newcommand{\thesisAuthorInitials}{\todo{А.\,В.}}
\newcommand{\thesisAuthor}             % Диссертация, ФИО автора
{%
    \texorpdfstring{% \texorpdfstring takes two arguments and uses the first for (La)TeX and the second for pdf
        \thesisAuthorLastName~\thesisAuthorOtherNames% так будет отображаться на титульном листе или в тексте, где будет использоваться переменная
    }{%
        \thesisAuthorLastName, \thesisAuthorOtherNames% эта запись для свойств pdf-файла. В таком виде, если pdf будет обработан программами для сбора библиографических сведений, будет правильно представлена фамилия.
    }
}
\newcommand{\thesisAuthorShort}        % Диссертация, ФИО автора инициалами
{\thesisAuthorInitials~\thesisAuthorLastName}
%\newcommand{\thesisUdk}                % Диссертация, УДК
%{\todo{xxx.xxx}}
\newcommand{\thesisTitle}              % Диссертация, название
{\todo{Исследование производительности высокопроизводительных вычислительных систем}}
\newcommand{\thesisSpecialtyNumber}    % Диссертация, специальность, номер
{\todo{05.13.15}}
\newcommand{\thesisSpecialtyTitle}     % Диссертация, специальность, название
{\todo{Вычислительные машины, комплексы и компьютерные сети}}
\newcommand{\thesisDegree}             % Диссертация, ученая степень
{\todo{доктора технических наук}}
\newcommand{\thesisDegreeShort}        % Диссертация, ученая степень, краткая запись
{\todo{д.т.н.}}
\newcommand{\thesisCity}               % Диссертация, город написания диссертации
{\todo{Новосибирск}}
\newcommand{\thesisYear}               % Диссертация, год написания диссертации
{\todo{2019}}
\newcommand{\thesisOrganization}       % Диссертация, организация
{\todo{ Министерство науки и высшего образования Российской Федерации  \\ Федеральное государственное бюджетное учреждение науки \\Институт вычислительной математики и математической геофизики \\ Сибирского отделения Российской академии наук}}
\newcommand{\thesisOrganizationShort}  % Диссертация, краткое название организации для доклада
{\todo{ИВМиМГ СО РАН}}

\newcommand{\thesisInOrganization}     % Диссертация, организация в предложном падеже: Работа выполнена в ...
{\todo{Федеральном государственном бюджетном учреждении науки Институте Вычислительной Математики и Математической Геофизики Сибирского отделения Российской академии наук}}

\newcommand{\supervisorFio}            % Научный руководитель, ФИО
{\todo{Лаврентьев Михаил Михайлович}}
\newcommand{\supervisorRegalia}        % Научный руководитель, регалии
{\todo{доктор физико-математических наук, профессор}}
\newcommand{\supervisorFioShort}       % Научный руководитель, ФИО
{\todo{М.\,М.~Лаврентьев}}
\newcommand{\supervisorRegaliaShort}   % Научный руководитель, регалии
{\todo{д.ф.-м.н., проф.}}


\newcommand{\opponentOneFio}           % Оппонент 1, ФИО
{\todo{Воеводин Владимир Валентинович}}
\newcommand{\opponentOneRegalia}       % Оппонент 1, регалии
{\todo{доктор физико-математических наук}}
\newcommand{\opponentOneJobPlace}      % Оппонент 1, место работы
{\todo{член.-корр. РАН}}
\newcommand{\opponentOneJobPost}       % Оппонент 1, должность
{\todo{зам.директора НИВЦ МГУ}}

\newcommand{\opponentTwoFio}           % Оппонент 2, ФИО
{\todo{Калайда Владимир Тимофеевич}}
\newcommand{\opponentTwoRegalia}       % Оппонент 2, регалии
{\todo{доктор технических наук}}
\newcommand{\opponentTwoJobPlace}      % Оппонент 2, место работы
{\todo{ТГУ}}
\newcommand{\opponentTwoJobPost}       % Оппонент 2, должность
{\todo{профессор}}

\newcommand{\opponentThreeFio}           % Оппонент 3, ФИО
{\todo{Рояк Михаил Эммануилович}}
\newcommand{\opponentThreeRegalia}       % Оппонент 3, регалии
{\todo{доктор технических наук}}
\newcommand{\opponentThreeJobPlace}      % Оппонент 3, место работы
{\todo{НГТУ}}
\newcommand{\opponentThreeJobPost}       % Оппонент 3, должность
{\todo{профессор}}



\newcommand{\leadingOrganizationTitle} % Ведущая организация, дополнительные строки
{\todo{Институт Гидродинамики им. М.А.Лаврентьева Сибирского отделения Российской академии наук}}

\newcommand{\defenseDate}              % Защита, дата
{\todo{DD mmmmmmmm YYYY~г.~в~XX часов}}
\newcommand{\defenseCouncilNumber}     % Защита, номер диссертационного совета
{\todo{Д\,219.005.02}}
\newcommand{\defenseCouncilTitle}      % Защита, учреждение диссертационного совета
{\todo{при федеральном государственном бюджетном образовательном учреждении высшего образования «Сибирский государственный университет телекоммуникаций и информатики»}}
\newcommand{\defenseCouncilAddress}    % Защита, адрес учреждение диссертационного совета
{\todo{630102, г. Новосибирск, ул. Кирова, д. 86, ауд. 625}}
\newcommand{\defenseCouncilPhone}      % Телефон для справок
{\todo{+7 (383) 269-82-75}}

\newcommand{\defenseSecretaryFio}      % Секретарь диссертационного совета, ФИО
{\todo{Нечта Иван Васильевич}}
\newcommand{\defenseSecretaryRegalia}  % Секретарь диссертационного совета, регалии
{\todo{канд. техн. наук}}            % Для сокращений есть ГОСТы, например: ГОСТ Р 7.0.12-2011 + http://base.garant.ru/179724/#block_30000

\newcommand{\synopsisLibrary}          % Автореферат, название библиотеки
{\todo{СибГУТИ и на сайте: http://www.sibsutis.ru/science/postgraduate/dis_sovets}}
\newcommand{\synopsisDate}             % Автореферат, дата рассылки
{\todo{DD mmmmmmmm YYYY года}}

% To avoid conflict with beamer class use \providecommand
\providecommand{\keywords}%            % Ключевые слова для метаданных PDF диссертации и автореферата
{}
