%% Согласно ГОСТ Р 7.0.11-2011:
%% 5.3.3 В заключении диссертации излагают итоги выполненного исследования, рекомендации, перспективы дальнейшей разработки темы.
%% 9.2.3 В заключении автореферата диссертации излагают итоги данного исследования, рекомендации и перспективы дальнейшей разработки темы.
в диссертации предложена и реализована оригинальная методика комплексного тестирования мультиархитектурных параллельных вычислительных систем, основанная на используемой в реальных расчетах программе для математического моделирования, а именно на реализации метода частиц в ячейках.
\begin{enumerate}
	\item Создан программный комплекс, основанный на моделировании динамики плазмы методом частиц в ячейках на высокопроизводительных ВС для всестороннего исследования производительности ВС, позволяющий в рамках одного запуска программы определить конкретную подсистему, наиболее заметно снижающую производительность. 
	
	\item Реализован и протестирован метод исследования коммуникационной структуры высокопроизводительных ВС, позволяющий давать рекомендации по более оптимальному распределению процессов в приложении на узлах высокопроизводительной ВС, а также вычислять экcтраполяцию реально полученной производительности на аналогичные системы с большим количеством узлов и процессоров.
	
	\item Разработана и обоснована методика расчета абсолютной оценки пригодности данной ВС для решения реальных задач, основанной на сбалансированности производительности различных подсистем конкретной ВС, в частности оперативной памяти, коммуникационной сети, дисковой подсистемы, процессоров и ускорителей вычислений и позволяющей сравнивать ВС безотносительно используемых программ и решаемых задач. 
	
	\item Предложен и протестирован метод комплексного анализа производительности узлов мультиархитектурной ВС, оснащенной многоядерными процессорами и графическими ускорителями или ускорителями Intel Xeon Phi, основанный на программе для моделирования динамики плазмы методом частиц в ячейках и позволяющий делать прогнозы эффективности данной мультиархитектурной ВС для решения конкретных задач, более достоверные по сравнению с синтетическими тестами.
\end{enumerate}


\textbf{Рекомендации и перспективы дальнейшей разработки темы.}
Основным направлением совершенствования разработанного теста является автоматическая выработка рекомедаций по оптимизации кода, охватывающих не только метод частиц в ячейкаx?  но и остальные наиболее часто используемые в математическом моделировании методы,  под конкретную протестированную ВС.

Одним из наиболее важных вариантов дальнейшего развития программы-теста, созданного в диссертационной работе является перенос на не охваченные в текущем варианте платформы, такие как Android, процессоры архитектуры Sunway, графические ускорители, использующие технологию OpenCL, в частности, AMD Firestream, а также на программируемые логические интегральные схемы (ПЛИС).

Кроме того, необходимостью является апробация теста на крупных высокопроизводительных ВС мощностью более петафлопса, а также - возможно, после специальной адаптации схемы расчета движения частиц - на ВС векторной архитектуры.
