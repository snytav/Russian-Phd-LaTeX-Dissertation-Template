%% Согласно ГОСТ Р 7.0.11-2011:
%% 5.3.3 В заключении диссертации излагают итоги выполненного исследования, рекомендации, перспективы дальнейшей разработки темы.
%% 9.2.3 В заключении автореферата диссертации излагают итоги данного исследования, рекомендации и перспективы дальнейшей разработки темы.
В работе предложена и реализована оригинальная методика комплексного тестирования мультиархитектурных параллельных вычислительных систем, основанная на используемой в реальных расчетах программе для математического моделирования. а именно на реализации метода частиц в ячейках.

Особенностями предложенной методики комплексного тестирования являются возможность определения для конкретной ВС абсолютной оценки, основанной на степени пригодности данной ВС для решения задач математического моделирования, метод измерения возрастания потока данных в коммуникационной сети ВС, а также оценка эффективности реализации вычислительных алгоритмов для мультиархитектурных ВС.

\textbf{Рекомендации и перспективы дальнейшей разработки темы.}
Основным направлением совершенствования разработанного теста является автоматическая выработка рекомедаций по оптимизации кода, охватывающих не только метод частиц в ячейках но и остальные наиболее часто используемые в математическом моделировании методы,  под конкретную протестированную ВС.

Одним из наиболее важных вариантов дальнейшего развития программы-теста, созданного в диссертационной работе является перенос на не охваченные в текущем варианте платформы, такие как Android, процессоры архитектуры Sunway, графические ускорители, использующие технологию OpenCL, в частности, AMD Firestream, а также на программируемые логические интегральные схемы (ПЛИС).

Кроме того, необходимостью является апробация теста на крупных высокопроизводительных ВС мощностью более петафлопса, а также - возможно, после специальной адаптации схемы расчета движения частиц - на ВС векторной архитектуры.
