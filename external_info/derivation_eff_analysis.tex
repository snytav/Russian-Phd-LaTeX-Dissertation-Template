\documentclass{book}
\usepackage{amsmath}

\usepackage{graphicx}
\usepackage{listings}

\usepackage{caption}
\usepackage{subcaption}

\usepackage[utf8]{inputenc}
\usepackage[english,russian]{babel}

\begin{document}
	\begin{equation}
	\label{weak_eff}
	\eta^{weak}_N = \frac{T^1(1)}{T^N(N)}
	\end{equation}
	здесь $T^K(N)$ обозначает время счета задачи с характерной размерностью $K$ при использовании $N$ процессоров.
	это время состоит из двух основных частей:
	\begin{itemize}
		\item собственно времени счета $T^K_{P}(N)$;
		\item времени коммуникаций $T^K_C(N)$;
	\end{itemize}
	Таким образом,
	\begin{equation}
	\label{weak_eff_detailed}
	\eta^{weak}_N = \frac{T^1(1)}{T^N_{P}(N)+T^N_C(N)}
	\end{equation}
	при выполнении вычислений одновременно на однотипных процессорах
	можно считать, что $T^1(1) = T^N_C(N)$, таким образом формула
	\ref{weak_eff_detailed} приводится к виду:
	\begin{equation}
	\label{weak_eff_detailed-time}
	\eta^{weak}_N = \frac{1}{1+ \frac{T^N_{C}(N)}{T^1(1)}}.
	\end{equation}
	Это означает, что при известном времени расчета задачи на одном процессоре можно восстановить время коммуникаций по эффективности в слабом смысле:
	\begin{equation}
	\label{comm_time_from_efficiency}
	T^N_{C}(N) = T^1(1) \left(\frac{1}{\eta^{weak}_N} - 1\right)
	\end{equation}
	далее время коммуникаций может быть источником для вычисления производительности коммуникационной сети при известном объеме пересылок.
	
	Что касается масштабируемости как интегральной характеристики ВС,то хорошая масштабируемость на коллективных, очевидно нелинейных коммуникациях является важным показателем связности (сохранения произв. сети при увеличении ее размера) ВС и пригодности ее к решению одной большой задачи (\textbf{придумать здесь градуировку}).
	
	
\end{document}