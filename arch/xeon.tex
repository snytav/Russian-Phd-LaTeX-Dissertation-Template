\documentclass[a4paper,14pt]{article}
\usepackage[english,russian]{babel}
\usepackage[utf8]{inputenc}


\begin{document}

\begin{table}[ht]
\begin{tabular}{|c|c|c|c|c|c|c|}
\hline
Номер процессора & E5502 & E5504 & E5506 & E5520 & E5530 & E5540 \\ \hline
Количество ядер  & 2 & \multicolumn{5}{c|}{4} \\ \hline
Количество потоков & 2 & \multicolumn{2}{c|}{4} & \multicolumn{3}{c|}{8} \\ \hline
Тактовая частота   &          &       &          &          &          &  \\
 процессора        & 1,86 ГГц & 2 ГГц & 2,13 ГГц & 2,26 ГГц & 2,5 ГГц & 2,53 ГГц \\ \hline
Кэш-память         & \multicolumn{3}{c|}{4 МБ «умный» кэш} & \multicolumn{3}{c|}{8 МБ «умный» кэш} \\ \hline
Производительность &  \multicolumn{3}{c|}{}          & \multicolumn{3}{c|}{} \\   
системной шины     & \multicolumn{3}{c|}{4,8 ГТ/сек} & \multicolumn{3}{c|}{5,86 ГТ/сек} \\ \hline
Набор команд       &\multicolumn{6}{c|}{64-битный} \\ \hline
Тепловыделение     & \multicolumn{6}{c|}{80 Вт} \\ \hline
Диапазон напряжения & \multicolumn{6}{c|}{}  \\
питания, VID        & \multicolumn{3}{c|}{0,75—1,35 В} \\ \hline
Макс. объём памяти  & \multicolumn{6}{c|}{}  \\
(зависит от типа памяти) & \multicolumn{6}{c|}{144 ГБ} \\ \hline
Типы памяти & \multicolumn{2}{c|}{DDR3-800} & \multicolumn{2}{c|}{DDR3-800/1066} & \multicolumn{2}{c|}{DDR3-800/1066} \\ \hline
Количество каналов & \multicolumn{6}{c|} \\ \hline
памяти & \multicolumn{6}{c|}{3} \\ \hline
%Макс. пропускная способность памяти & 19,2 ГБ/сек & 25,6 ГБ/сек \\.
%Расширение физического адреса & 40-битовое \\.
%Поддержка памятью функции \href{https://ru.wikipedia.org/wiki/ECC}{ECC} & Да & Да & Да & Да & Да & Да \\.
%\textbf{Спецификация корпуса} \\.
%Макс. процессоров в конфигурации & 2 \\.
%Температура корпуса &  & 76\nolinebreak°C \\.
%Размер корпуса & 42,5×45 мм \\.
%Норма \href{https://ru.wikipedia.org/wiki/%D0%A4%D0%BE%D1%82%D0%BE%D0%BB%D0%B8%D1%82%D0%BE%D0%B3%D1%80%D0%B0%D1%84%D0%B8%D1%8F}{литографии}\href{https://ru.wikipedia.org/wiki/%D0%A2%D0%B5%D1%85%D0%BF%D1%80%D0%B
%Размер ядра процессора & 263 мм² \\.
%Количество транзисторов в ядре & 731 млн \\.
%Поддержка \href{https://ru.wikipedia.org/wiki/%D0%A0%D0%B0%D0%B7%D1%8A%D1%91%D0%BC_%D0%BF%D1%80%D0%BE%D1%86%D0%B5%D1%81%D1%81%D0%BE%D1%80%D0%B0_%D0%BF%D0%B5%D1%80%D1%81%D0%BE%D0%BD%D0%B0%D0%BB%D1%8C%D0%BD%D0%BE
%Без\href{https://ru.wikipedia.org/wiki/%D0%93%D0%B0%D0%BB%D0%BE%D0%B3%D0%B5%D0%BD%D1%8B}{галогенная} продукция? & Да \\.
%\textbf{Применяемые технологии} \\.
%\href{https://ru.wikipedia.org/wiki/Turbo_Boost}{Turbo Boost} & Нет & Да \\.
%\href{https://ru.wikipedia.org/wiki/Hyper-Threading}{Hyper-Threading} & Нет & Нет & Нет & Да & Да & Да \\.
%\href{https://ru.wikipedia.org/wiki/%D0%92%D0%B8%D1%80%D1%82%D1%83%D0%B0%D0%BB%D0%B8%D0%B7%D0%B0%D1%86%D0%B8%D1%8F}{Виртуализация} (VT-x) & Да \\.
%Виртуализации прямого ввода-вывода (\href{https://ru.wikipedia.org/wiki/VT-d}{VT-d}) & Да \\.
%\href{https://ru.wikipedia.org/wiki/%D0%94%D0%BE%D0%B2%D0%B5%D1%80%D0%B5%D0%BD%D0%BD%D0%B0%D1%8F_%D0%B7%D0%B0%D0%B3%D1%80%D1%83%D0%B7%D0%BA%D0%B0_(%D0%B0%D0%BF%D0%BF%D0%B0%D1%80%D0%B0%D1%82%D0%BD%D1%8B%D0%B5_%D
%Новые AES инструкции &  \\.
%Intel 64 & Да \\.
%Idle States & Да \\.
%Улучшенная технология \href{https://ru.wikipedia.org/wiki/SpeedStep}{SpeedStep} & Да \\.
%Demand Based Switching & Да \\.
%\href{https://ru.wikipedia.org/wiki/NX-%D0%B1%D0%B8%D1%82}{Execute Disable Bit} & Да
%\end{tabular}
\end{tabular}
\end{table}

\end{document}