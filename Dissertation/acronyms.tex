\chapter*{Список сокращений и условных обозначений} % Заголовок
\addcontentsline{toc}{chapter}{Список сокращений и условных обозначений}  % Добавляем его в оглавление
\noindent
%\begin{longtabu} to \dimexpr \textwidth-5\tabcolsep {r X}
\begin{longtabu} to \textwidth {r X}
% Жирное начертание для математических символов может иметь
% дополнительный смысл, поэтому они приводятся как в тексте
% диссертации

\textbf{MPI} & Message Passing Interface \\
\textbf{CUDA} & Compute Uniform Device Architecture\\
\textbf{FDTD} & finite difference time domain, метод конечных
разностей во~временной области\\
\textbf{PIC} & Particle-In-Cell,  метод частиц в ячейках\\
\textbf{GPU} & Graphical Processor Unit, графический процессор\\
\textbf{PE} & Processor Element, процессорный элемент\\
\textbf{ВС} & Вычислительная система\\
\textbf{ВВС} & Высокопроизводительная вычислительная система\\
\textbf{ПЭ} & процессорный элемент\\
\textbf{FLOPS} & Floating Point Opearations Per Second, количество операций с плавающей точкой в секунду\\
\textbf{GFLOPS} & billions of Floating Point Opearations Per Second, количество миллиардов операций с плавающей точкой в секунду\\
\textbf{GB} & gygabyte, гигабайт\\



\end{longtabu}
\addtocounter{table}{-1}% Нужно откатить на единицу счетчик номеров таблиц, так как предыдующая таблица сделана для удобства представления информации по ГОСТ
