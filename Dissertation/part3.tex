\chapter{Комплексная оценка производительности ВС}
В \textbf{третьей главе} описана методика измерения характеристик ВС с помощью программы, релизующей метод частиц в ячейках.


\textbf{ТАБЛИЦУ С НАМЕРЯННЫМИ ЗНАЧЕНИЯМИ на основе статьи ПАВТ10
везде написать пиковые знач. - Курносов 
}
Предложена методика комплексной оценки тестируемой ВС с точки зрения возможности эффективной реализзации математических моделей на основе определения баланса между скоростью счета и скоростью пересылки данных между узлами ВС. Баланс определяется на основе усреднения данных расчетов по методу частиц в ячейках, который используется в качестве оценки снизу по скорости счета и оценки сверху по памяти для большинства существующих математических методов.

Кроме того, на основе проведенных расчетов измерена скорость счета и скорость перемещения данных для нескольких протестированных ВС.   

%		В настоящий момент списки Top50 и Top500
%		выстроены в порядке убывания пиковой производительности и производительности на тесте LinPack что, разумеется, дает определенную информацию
%		о сравнительной скорости работы представленных там машин. Но очень многие факторы, такие как скорость работы и объем дисков, пропускная 
%		способность шины памяти и коммуникационной сети, неоднородность оборудования и т.д. - остаются за пределами рассмотрения. А это именно те 
%		проблемы, с которыми придется столкнуться при попытке посчитать на кластере большую задачу. По этой причине тестирование продится с помью измерения времени, затрачитваемого на различные этапы программы, рещающей реальную физическую задачу.





%	Общие идеи - на основе статьи ПаВТ-2010
%	Перемерять на: Ломоносов, МВС-10П новый раздел, НКС-1П, Политехник (tornado), IBM (Хабаровск)
\section{Расчет производительности процессорных элементов}
\label{calc_PE}
В \textit{первом разделе} описана методика измерения производительности процессорных элементов.
Для того, чтобы отделить время счета от времени обращения к оперативной памяти было рассмотрено время работы процедуры,
реализующей одномерное преобразование Фурье, которая является частью физической диагностики, используемой в при моделировании динамики плазмы. Измереннное время с учетом известного размера данных и и количества операций в БПФ (\textit{Е.П.Овсянников и др.}), переводится во флопсы. Сравнительная производительность процессорных элементов некоторых из рассмотренных в диссертационной работе ВС выглядит как показано на рисунке  \ref{procs_flops}:

\begin{figure}[htb]
	\begin{center}
		\includegraphics[height=7cm,keepaspectratio]{images/processor_FLOPS.png}
	\end{center}
	\caption{Производительность процессоров Intel Xeon, измеренная в ходе выполнения одномерного преобразования Фурье на некоторых кластерах. Размерность преобразования $N=64$. Измерения выполнены в 2010 г.}
	\label{procs_flops}
\end{figure} 




\section{Расчет производительности системы памяти}
Во \textit{втором разделе} описано измерение производительности системы памяти, основанное на измерении времени расчета движения модельных частиц, при этом благодаря алгоритмически особенностям метода частиц в ячейках удается исключить использользование кэш-памяти и производить измерение скорости доступа именно к оперативной памяти.

Переход от фактически измеренной величины, времени выполнения расчета движения модельных частиц выполнялся из следующих соображений: на каждое из используемых ядер приходится 2.5 млн. модельных частиц, каждая частица занимает 48 байт, кроме того, для расчета движения частицы необходимы значения электрического и магнитного полей в той ячейке сетки, где находится частица. Это означает, что для каждой из 6 компонент электромагнитного поля загружается 8 значений, соответствующих вершинам параллелепипеда, то есть ячейки сетки. 

Более того, по результатам расчета движения модельной частицы вычисляется вклад данной частицы в ток. Для каждой из трех меняются значения в 4 узлах сетки компоненты тока, которые вместе с новыми значениями координаты и импульса модельной частицы сохраняются в оперативную память.

Таким образом для каждой модельной частицы загружается из памяти 432 байта и сохраняется 144 байта, общий поток данных составляет 576 байт на одну частицу.

\textbf{в третью главу - раздел про перенос на Phi с текстами}


\begin{figure}[htb]
	\begin{center}
		\includegraphics[height=7cm,keepaspectratio]{images/data_load_GBsec.png}
	\end{center}
	\caption{Скорость загрузки данных из оперативной памяти на этапе расчета движения модельных частиц на некоторых кластерах. Количество модельных частиц: 2.5 млн. на каждое процессорное ядро. Измерения выполнены в 2010 г.}
	\label{PIC_RAM}
\end{figure}
Следует отметить, что вопрос о сравнении чисел на рис. \ref{PIC_RAM} с техническими характеристиками шины памяти 
является второстепенным. Основной вопрос в данном случае - это измерение скорости работы памяти фактически доступной для расчетного приложения.

\subsubsection{Расчет производительности коммуникационной сети}
В \textit{третьем разделе} приведена методика измерения быстродействия коммуникационной сети на основе анализа времени работы MPI-процедур, осуществляющих обмен граничными значениями между отдельными подобластями при решении уравнений Максвелла и при пересылке модельных частиц. В силу того, что при этом используются различные виды коммуникационных функций  - как блокирующие, так и не блокирующие, как парные, так и коллективные, при использовании эйлерово-лагранжевой декомпозиции - это позволяет набрать в течение одного расчета большую базу данных для получения знаний о структуре коммуникационной сети, времени прохождения сообщений в зависимоссти от размера, системных таймаутах и пр. 

%		На  рисунке приведено скорость обмена данными между процессами (график из статьи в Мб/сек) 
%		\textbf{перерисовать графики МВС-100К и Ломоносов в общей части}	
%		
%		
%		Сравнение графиков ускорения, полученных на МВС-100К, Ломоносов НГУ и пр.

\section{Формула для комплексной оценки ВС}
\label{complex_evaluation}
В \textit{четвертом разделе} приведено обоснование формулы, на основании которой выносится оценка ВС по материалам проведенных тестов. При этом важно отметить, что оценка является не сравнительной - относительно других ВС, а абсолютной - с точки зрения математического моделирования. 

В частности, для того, чтобы параллельная ВС могла быть признана адаптированной к задачам математического моделирования, она должна соответвовать следующим требованиям:
\begin{enumerate}
	\item Относительно высокая производительность коммуникационной сети, позволяющая пересылать все необходимые для расчета данные, не задерживая вычислений
	\item Очень высокая пропускная способность дисковой подсистемы, обеспечивающая сохранение больших объемов данных, полученных в результате счета  	
\end{enumerate}

Важно отметить, что названы относительные показатели, обеспечивающие возможность пересылать и сохранять данные, без ущерба для скорости вычислений. Именно это и означает  комплексную пригодность ВС к решению задач математического моделирования, т.е. результаты счета сохраняются на диск с той же скоростью, с которой пересылаются данные между узлами данной ВС, и более того, эта скорость не намного меньше скорости вычислений.

Для того, чтобы все три упомянутые величины могли быть использованы в одной формуле, необходимо 
\begin{itemize}
	\item привести эти величины к одной размерности (скорость вычислений выражается во флопсах, скорость обмена данными - в гигабайтах в секунду)
	\item представить обобщенные коэффициенты, позволяющие сравнивать объем данных, сохраняемых на диск и объем данных, пересылаемых по коммуникационной сети ВС  
\end{itemize}


Для решения обоих этих задач использованы усредненные данные расчетов по методу частиц в ячейках на различных ВС. Данный метод может быть использован как оценка снизу, т.е. пригодность некоторой ВС для проведения расчетов по методу частиц в ячейках может трактоваться как возможность проведения расчетов по широкому спектру вычислительных методов, причем, как правило, с большей эффективностью.

Итак, коэффициент перевода из флопсов в байты в секунду для расчетов с частицами равен
$k_{f2b} = 500/576$ = 0.86   
и коээфициент для перевода объема данных, сохраняемых на диск к объему данных, пересылаемых по коммуникационной сети, аналогично, для частиц равен (усредненно):
$k_{MPI} = 0.05$ 
Это объясняется тем, что в среднем не более 5\% частиц пересылается между подобластями.
В итоге формула оценки $\xi$ имеет вид:
$$
\xi = \frac{W_{MPI}} {k_{f2b} W_{PIC}}, 
$$
при условии, что $W_{disc} \approx W_{MPI}$,
здесь $W_{disc}$ - скорость работы дисковой подсистемы (байт/сек), $W_{MPI}$
- скорость пересылки данных по сети - (байт/сек) и $W_{PIC}$ - скорость расчета по частицам (во флопсах).	

