\chapter{Основные характеристики процессоров и GPU, использованных в тестовых расчетах} \label{AppendixA}

\section{Процессоры семейства Intel Xeon}
\subsection{Устаревшие процессоры семейства Intel Xeon}

\begin{center}
\begin{table}[ht]
\caption{Процессоры Intel Xeon с индексами E}
	\begin{tabular}{|c|c|c|c|c|c|c|}
		\hline
		Номер процессора & E5502 & E5504 & E5506 & E5520 & E5530 & E5540 \\ \hline
		Количество ядер  & 2 & \multicolumn{5}{c|}{4} \\ \hline
		Количество потоков & 2 & \multicolumn{2}{c|}{4} & \multicolumn{3}{c|}{8} \\ \hline
		Тактовая частота   &          &       &          &          &          &  \\
		процессора, ГГц    & 1,86  & 2  & 2,13 & 2,26 & 2,5 & 2,53 \\ \hline
		Кэш-память         & \multicolumn{3}{c|}{4 МБ «умный» кэш} & \multicolumn{3}{c|}{8 МБ «умный» кэш} \\ \hline
		Производительность &  \multicolumn{3}{c|}{}          & \multicolumn{3}{c|}{} \\   
		системной шины     & \multicolumn{3}{c|}{4,8 ГТ/сек} & \multicolumn{3}{c|}{5,86 ГТ/сек} \\ \hline
		Набор команд       &\multicolumn{6}{c|}{64-битный} \\ \hline
		Тепловыделение     & \multicolumn{6}{c|}{80 Вт} \\ \hline
		Диапазон напряжения & \multicolumn{6}{c|}{}  \\
		питания, VID        & \multicolumn{6}{c|}{0,75—1,35 В} \\ \hline
		Макс. объём памяти  & \multicolumn{6}{c|}{}  \\
		(зависит от типа памяти) & \multicolumn{6}{c|}{144 ГБ} \\ \hline
		Типы памяти & \multicolumn{2}{c|}{DDR3-800} & \multicolumn{2}{c|}{DDR3-800/1066} & \multicolumn{2}{c|}{DDR3-800/1066} \\ \hline
		Количество каналов & \multicolumn{6}{c|}{} \\
		памяти & \multicolumn{6}{c|}{3} \\ \hline
		Макс. пропускная & \multicolumn{6}{c|}{} \\ \hline
		способность памяти & \multicolumn{3}{c|}{19,2 ГБ/сек} & \multicolumn{3}{c|}{25,6 ГБ/сек }\\ \hline
		Расширение         & \multicolumn{6}{c|}{} \\ 
		физического адреса & \multicolumn{6}{c|}{40-битовое} \\ \hline
		\multicolumn{7}{|c|}{Применяемые технологии} \\ \hline
		Turbo Boost & \multicolumn{3}{c|}{Нет} & \multicolumn{3}{c|}{Да} \\ \hline
		Hyper-Threading & Нет & Нет & Нет & Да & Да & Да \\ \hline
		Виртуализация (VT-x) & \multicolumn{6}{c|}{Да} \\ \hline
		Виртуализации прямого & \multicolumn{6}{c|}{} \\
		ввода-вывода (VT-d)   & \multicolumn{6}{c|}{ Да} \\ \hline
		Trusted Execution &   \multicolumn{6}{c|}{ Нет} \\ \hline
	\end{tabular}
\end{table}
\end{center}

Единицы измерения производительности системной шины - ГТ/с -

VID

VT-x

VT-d

Turbo Boost

Hyper-Threading

Trusted Execution



\subsubsection{Процессор Intel Xeon  E5540}
Данные этого подраздела в основном взяты с сайта \cite{geek_E5540}


\begin{table}[ht]
\begin{center}
\caption{Основные характеристики процессора Intel Xeon  E5540}
\begin{tabular}{|c|c|}
\hline	
Тактовая частота & 2.53GHz   \\ \hline
Количество ядер & 4 	     \\ \hline
кэш команд, L1 &  64 KB      \\ \hline
кэш данных, L1 &  64 KB       \\ \hline
кэш L2         &  512 KB      \\ \hline
кэш L3         &  0 KB        \\ \hline
\end{tabular}
\end{center} 	
\end{table} 	

\begin{table}[ht]
\begin{center}
\caption{Производительность процессора Intel Xeon  E5540 на операциях с плавающей точкой}
\begin{tabular}{|c|c|}
\hline	
Вычислительный алгоритм &  Производительность, GFLOPS \\ 
                 & в режиме многоядерных скалярных вычислений \\ \hline
Алгоритм Мандельброта  &  1.54 	\\ \hline
Скалярное произведение &  5.44   \\ \hline
LU-разложение          &  2.38   \\ \hline
Проверка простоты      & 0.568  \\ \hline 


		\end{tabular}
	\end{center} 	
\end{table} 	

\begin{table}[ht]
	\begin{center}
		\caption{Производительность памяти процессора Intel Xeon  E5540}
		\begin{tabular}{|c|c|}
			\hline	
			Операция  &  Производительность, GB/sec \\ \hline
			Последовательное чтение &  6.5 	\\  \hline
			Последовательная запись &  6.16   \\  \hline
			Выделение памяти средствами stdlib &  15.9 операций malloc в секунду  \\  \hline
			Запись средствами stdlib  & 7.64  \\ \hline
			Чтение средствами stdlib  & 7.53  \\ \hline 
		\end{tabular}
	\end{center} 	
\end{table} 	


%http://browser.geekbench.com/geekbench2/413456
\subsubsection{Intel Xeon 5150}

Данные этого подраздела в основном взяты с сайта \cite{geek_5150}


\begin{table}[ht]
	\begin{center}
		\caption{Основные характеристики процессора Intel Xeon  5150}
		\begin{tabular}{|c|c|}
			\hline	
			Тактовая частота & 2.72GHz   \\ \hline
			Количество ядер & 2 	     \\ \hline
			кэш команд, L1 &  0 KB      \\ \hline
			кэш данных, L1 &  0 KB       \\ \hline
			кэш L2         &  0 KB      \\ \hline
			кэш L3         &  0 KB        \\ \hline
		\end{tabular}
	\end{center} 	
\end{table} 	

\begin{table}[ht]
	\begin{center}
		\caption{Производительность процессора Intel Xeon  5150 на операциях с плавающей точкой}
		\begin{tabular}{|c|c|}
			\hline	
			Вычислительный алгоритм &  Производительность, GFLOPS \\ 
			& в режиме многоядерных скалярных вычислений \\ \hline
			Алгоритм Мандельброта  &  1.34 	\\ \hline
			Скалярное произведение &  2.12   \\ \hline
			LU-разложение          &  2.05   \\ \hline
			Проверка простоты      &  464  \\ \hline 
			
			
		\end{tabular}
	\end{center} 	
\end{table} 	

\begin{table}[ht]
	\begin{center}
		\caption{Производительность памяти процессора Intel Xeon 5150}
		\begin{tabular}{|c|c|}
			\hline	
			Операция  &  Производительность, GB/sec \\ \hline
			Последовательное чтение &  2.93 	\\  \hline
			Последовательная запись &  1.9   \\  \hline
			Выделение памяти средствами stdlib &  6.78 операций malloc в секунду  \\  \hline
			Запись средствами stdlib  & 2.03  \\ \hline
			Чтение средствами stdlib  & 1.96  \\ \hline 
		\end{tabular}
	\end{center} 	
\end{table} 	


\subsubsection{Процессор Intel Xeon E5472}
\begin{table}[ht]
	\begin{center}
		\caption{Основные характеристики процессора Intel Xeon  E5472}
		\begin{tabular}{|c|c|}
			\hline	
			Тактовая частота & 3.00GHz   \\ \hline
			Количество ядер & 4 	     \\ \hline
			кэш команд, L1 &  0 KB      \\ \hline
			кэш данных, L1 &  0 KB       \\ \hline
			кэш L2         &  6144 KB      \\ \hline
			кэш L3         &  0 KB        \\ \hline
		\end{tabular}
	\end{center} 	
\end{table} 	

\begin{table}[ht]
	\begin{center}
		\caption{Производительность процессора Intel Xeon  E5472 на операциях с плавающей точкой}
		\begin{tabular}{|c|c|}
			\hline	
			Вычислительный алгоритм &  Производительность, GFLOPS \\ 
			& в режиме многоядерных скалярных вычислений \\ \hline
			Алгоритм Мандельброта  &  6.203 	\\ \hline
			Скалярное произведение &  3.05   \\ \hline
			LU-разложение          &  2.7   \\ \hline
			Проверка простоты      &  13.9  \\ \hline 
			
			
		\end{tabular}
	\end{center} 	
\end{table} 	

\begin{table}[ht]
	\begin{center}
		\caption{Производительность памяти процессора Intel Xeon E5472}
		\begin{tabular}{|c|c|}
			\hline	
			Операция  &  Производительность, GB/sec \\ \hline
			Последовательное чтение &  2.7 	\\  \hline
			Последовательная запись &  3.4   \\  \hline
			Выделение памяти средствами stdlib &  6.78 операций malloc в секунду  \\  \hline
			Запись средствами stdlib  & 1.6  \\ \hline
			Чтение средствами stdlib  & 1.7  \\ \hline 
		\end{tabular}
	\end{center} 	
\end{table} 	



\subsubsection{Процессор Intel Xeon 5355}

\begin{table}[ht]
	\begin{center}
		\caption{Основные характеристики процессора Intel Xeon  5355}
		\begin{tabular}{|c|c|}
			\hline	
			Тактовая частота & 3.00GHz   \\ \hline
			Количество ядер & 4 	     \\ \hline
			кэш команд, L1 &  0 KB      \\ \hline
			кэш данных, L1 &  0 KB       \\ \hline
			кэш L2         &  4096 KB      \\ \hline
			кэш L3         &  0 KB        \\ \hline
		\end{tabular}
	\end{center} 	
\end{table} 	

\begin{table}[ht]
	\begin{center}
		\caption{Производительность процессора Intel Xeon  5355 на операциях с плавающей точкой}
		\begin{tabular}{|c|c|}
			\hline	
			Вычислительный алгоритм &  Производительность, GFLOPS \\ 
			& в режиме многоядерных скалярных вычислений \\ \hline
			Алгоритм Мандельброта  &  5.2 	\\ \hline
			Скалярное произведение &  2.6   \\ \hline
			LU-разложение          &  2.08   \\ \hline
			Проверка простоты      &  6.4  \\ \hline 
			
			
		\end{tabular}
	\end{center} 	
\end{table} 	

\begin{table}[ht]
	\begin{center}
		\caption{Производительность памяти процессора Intel Xeon 5355}
		\begin{tabular}{|c|c|}
			\hline	
			Операция  &  Производительность, GB/sec \\ \hline
			Последовательное чтение &  2.1 	\\  \hline
			Последовательная запись &  2.5   \\  \hline
			Выделение памяти средствами stdlib &  6.78 операций malloc в секунду  \\  \hline
			Запись средствами stdlib  & 1.36  \\ \hline
			Чтение средствами stdlib  & 1.4 \\ \hline 
		\end{tabular}
	\end{center} 	
\end{table} 	


\subsubsection{Intel Xeon 5570}

\begin{table}[ht]
	\begin{center}
		\caption{Основные характеристики процессора Intel Xeon  5570}
		\begin{tabular}{|c|c|}
			\hline	
			Тактовая частота & 2.93GHz   \\ \hline
			Количество ядер & 4 	     \\ \hline
			кэш команд, L1 &  0 KB      \\ \hline
			кэш данных, L1 &  0 KB       \\ \hline
			кэш L2         &  256 KB      \\ \hline
			кэш L3         &  0 KB        \\ \hline
		\end{tabular}
	\end{center} 	
\end{table} 	

\begin{table}[ht]
	\begin{center}
		\caption{Производительность процессора Intel Xeon  5570 на операциях с плавающей точкой}
		\begin{tabular}{|c|c|}
			\hline	
			Вычислительный алгоритм &  Производительность, GFLOPS \\ 
			& в режиме многоядерных скалярных вычислений \\ \hline
			Алгоритм Мандельброта  &  17.9 	\\ \hline
			Скалярное произведение &  29.2   \\ \hline
			LU-разложение          &  3.8  \\ \hline
			Проверка простоты      &  23.9  \\ \hline 
			
			
		\end{tabular}
	\end{center} 	
\end{table} 	

\begin{table}[ht]
	\begin{center}
		\caption{Производительность памяти процессора Intel Xeon 5570}
		\begin{tabular}{|c|c|}
			\hline	
			Операция  &  Производительность, GB/sec \\ \hline
			Последовательное чтение &  5.5 	\\  \hline
			Последовательная запись &  5.2   \\  \hline
			Выделение памяти средствами stdlib &  -  \\  \hline
			Запись средствами stdlib  & 3.9  \\ \hline
			Чтение средствами stdlib  & 8.1  \\ \hline 
		\end{tabular}
	\end{center} 	
\end{table} 	


\subsection{Современные процессоры семейства Intel Xeon}

\subsubsection{Intel Xeon E5-2697 v3}
\begin{table}[ht]
	\begin{center}
		\caption{Основные характеристики процессора Intel Xeon  E5-2697 v3}
		\begin{tabular}{|c|c|}
			\hline	
			Тактовая частота & 2.6GHz   \\ \hline
			Количество ядер & 28 	    \\ \hline
			кэш команд, L1 &  32 KB     \\ \hline
			кэш данных, L1 &  32 KB     \\ \hline
			кэш L2         &  256 KB    \\ \hline
			кэш L3         &  35 MB     \\ \hline
		\end{tabular}
	\end{center} 	
\end{table} 	

\begin{table}[ht]
	\begin{center}
		\caption{Производительность процессора Intel Xeon  5570 на операциях с плавающей точкой}
		\begin{tabular}{|c|c|}
			\hline	
			Вычислительный алгоритм &  Производительность, GFLOPS \\ 
			& в режиме многоядерных скалярных вычислений \\ \hline
		    LinPACK  &  2.99 	\\ \hline
			Теоретическая пиковая &  700   \\ \hline
		\end{tabular}
	\end{center} 	
\end{table} 	

\subsubsection{Intel XeonE5-2697Av4}

\begin{table}[ht]
	\begin{center}
		\caption{Основные характеристики процессора Intel Xeon  E5-2697Av4}
		\begin{tabular}{|c|c|}
			\hline	
			Тактовая частота & 2.6GHz   \\ \hline
			Количество ядер & 16 	     \\ \hline
			Мощность        &  145 W      \\ \hline
			Происводительность (пиковая) &  700 GFLOPS       \\ \hline
			<<умный>> кэш   &  45 MB      \\ \hline
			кэш L3          &  0 KB        \\ \hline
		\end{tabular}
	\end{center} 	
\end{table} 	





\section{Процессор Phenom II X6 1055T}
\begin{table}[ht]
	\begin{center}
		\caption{Основные характеристики процессора Intel Xeon  5570}
		\begin{tabular}{|c|c|}
			\hline	
			Тактовая частота & 2.80GHz   \\ \hline
			Количество ядер & 6 	     \\ \hline
			кэш команд, L1 &  64 KB      \\ \hline
			кэш данных, L1 &  64 KB       \\ \hline
			кэш L2         &  512 KB      \\ \hline
			кэш L3         &  6144 KB        \\ \hline
		\end{tabular}
	\end{center} 	
\end{table} 	

\begin{table}[ht]
	\begin{center}
		\caption{Производительность процессора Intel Xeon  5570 на операциях с плавающей точкой}
		\begin{tabular}{|c|c|}
			\hline	
			Вычислительный алгоритм &  Производительность, GFLOPS \\ 
			& в режиме многоядерных скалярных вычислений \\ \hline
			Алгоритм Мандельброта  &  12.9 	\\ \hline
			Скалярное произведение &  18.1   \\ \hline
			LU-разложение          &  12.2  \\ \hline
			Проверка простоты      &  14.5  \\ \hline 
			
			
		\end{tabular}
	\end{center} 	
\end{table} 	

\begin{table}[ht]
	\begin{center}
		\caption{Производительность памяти процессора Intel Xeon 5570}
		\begin{tabular}{|c|c|}
			\hline	
			Операция  &  Производительность, GB/sec \\ \hline
			Последовательное чтение &  3.4 	\\  \hline
			Последовательная запись &  3.8   \\  \hline
			Выделение памяти средствами stdlib &  -  \\  \hline
			Запись средствами stdlib  & 2.8  \\ \hline
			Чтение средствами stdlib  & 3.3  \\ \hline 
		\end{tabular}
	\end{center} 	
\end{table} 	


\clearpage
\section{Графические ускорители (GPU)}

\begin{table}[ht]
	\begin{center}
		\caption{Основные параметры GPU, использованных в тестах}
		\begin{tabular}{|c|c|c|c|}
			\hline
			Название                &  GeForce GTX 850M & Tesla M2090 & Tesla K40m \\ \hline
			Объем                  &                   &             &             \\
			глобальной памяти       & 4096 MB           & 5375 MB     & 11520 MB \\ \hline
			Количество              &             &               &     \\
			мультипроцессоров       & 5           &  16           & 15  \\ \hline
			Количество              &             &               &     \\
			ядер CUDA               & 640         & 512         & 2880  \\ \hline
			Тактовая частота        &             &             &         \\
			GPU                     & 902 MHz     & 1301 MHz    & 745 MHz \\ \hline
			Ширина                  &             &             &         \\
			шины памяти             & 128-bit     & 384-bit     & 384-bit \\ \hline
		\end{tabular}
		\label{GPUs}
	\end{center}
\end{table}



\chapter{Краткое описание основных свойств сетевого оборудования и сетевых протоколов, использованных в тестовых расчетах} \label{AppendixB}

\section{Gigabit Ethernet}

Gigabit Ethernet (GE, GbE, или 1 GigE) в компьютерных сетях — термин, описывающий различные технологии передачи Ethernet-кадров со скоростью 1 гигабит в секунду, определяемые рядом стандартов группы IEEE 802.3. Используется для построения проводных локальных сетей с 1999 года, постепенно вытесняя Fast Ethernet благодаря значительно более высокой скорости передачи данных. При этом необходимые кабели и часть сетевого оборудования мало отличаются от используемых в предыдущих стандартах, широко распространены и обладают низкой стоимостью.

Ранее в стандарте описывались полудуплексные гигабитные соединения с использованием сетевых концентраторов[1], но эта спецификация больше не обновляется, и сейчас используется исключительно полнодуплексный режим с соединением через коммутаторы. 


Всего существует пять стандартов физического уровня для гигабитного Ethernet, использующих оптоволоконный кабель (1000BASE-X), витую пару (1000BASE-T) или экранированный сбалансированный медный кабель (1000BASE-CX).

Стандарт IEEE 802.3z включает в себя 1000BASE-SX для передачи сигнала по многомодовому оптоволокну, 1000BASE-LX — по одномодовому оптоволокну, и почти вышедший из употребления 1000BASE-CX — по экранированному сбалансированному медному кабелю. Эти стандарты используют кодирование 8b/10b, которое повышает скорость передачи линии на 25 %, с 1000 Мбит/с до 1250 Мбит/с. Символы затем отправляются с использованием кода NRZ.

IEEE 802.3ab, в котором описан широко распространённый тип интерфейса 1000BASE-T, использует другую схему кодирования, чтобы поддерживать скорость передачи символов на как можно более низком уровне для отправки данных по витой паре.

IEEE 802.3ap определяет работу Ethernet на электронных объединительных платах при различных скоростях.

Ethernet in the First Mile позднее добавил стандарты 1000BASE-LX10 и -BX10. 

\section{Infiniband}

Infiniband (иногда сокр. IB) — высокоскоростная коммутируемая компьютерная сеть, используемая в высокопроизводительных вычислениях, имеющая очень большую пропускную способность и низкую задержку. Также используется для внутренних соединений в некоторых вычислительных комплексах. По состоянию на 2014 год Infiniband являлся наиболее популярной сетью для суперкомпьютеров. Контроллеры Infiniband (host bus adapter) и сетевые коммутаторы производятся компаниями Mellanox и Intel. При создании Infiniband в него закладывалась масштабируемость, сеть использует сетевую топологию на основе коммутаторов (Switched fabric).

В качестве коммуникационной сети кластеров Infiniband конкурирует с группой стандартов Ethernet и проприетарными технологиями[1], например, компаний Cray и IBM. При построении компьютерных сетей IB конкурирует с Gigabit Ethernet, 10 Gigabit Ethernet, и 40/100 Gigabit Ethernet. Также IB используется для подключения накопителей информации DAS.[2] Развитием и стандартизацией технологий Infiniband занимается InfiniBand Trade Association (англ.)русск.[3]. 

Подобно многим современным шинам, например, PCI Express, SATA, USB 3.0, в Infiniband используются дифференциальные пары для передачи последовательных сигналов. Две пары вместе составляют одну базовую двунаправленную последовательную шину (англ. lane), обозначаемую 1х. Базовая скорость — 2,5 Гбит/с в каждом направлении. Порты Infiniband состоят из одной шины или агрегированных групп 4x или 12x базовых двунаправленных шин. Чаще всего применяются порты 4x[4].

Для портов существует несколько режимов передачи данных по шинам. Более ранние режимы использовали для балансировки сигнала кодирование 8B/10B[5] (каждые 8 бит данных передаются по шине как 10 бит) с накладными расходами в 20 %:

Single Data Rate (SDR, 1999) — работа с базовой скоростью 2,5 Гбит/с, эффективная скорость (с учетом расходов на кодирование) 2 Гбит/с на каждую шину
Double Data Rate (DDR, 2004) — битовая скорость равна удвоенной базовой (5 Гбит/с, эффективная 4 Гбит/с). 4x порт имеет физическую скорость 20 Гбит/с и эффективную 16 Гбит/с
Quad Data Rate (QDR, 2008) — соответственно, учетверенной (базовая 10 Гбит/с), эффективная для 4x порта 32 Гбит/с.

Начиная с режима FDR-10 применяется намного более экономичное кодирование 64B/66B:

Fourteen Data Rate 10 (FDR-10) — эффективная скорость на 1x шину чуть более 10 Гбит/с, для 4x порта 40 Гбит/с
Fourteen Data Rate (FDR, 2011) — базовая скорость 1х шины 14.0625 Гбит/с[6], 4x порт предоставляет около 56 Гбит/с
Enhanced Data Rate (EDR) — скорость 1x 25.78125 Гбит/с, 4x — около 100 Гбит/с

Основное назначение Infiniband — межсерверные соединения, в том числе и для организации RDMA (Remote Direct Memory Access). 


\section{Omnipath}

Intel Omni-Path (иногда сокр. Intel OPA) — высокопроизводительная коммуникационная архитектура от компании Intel, предназначенная для высокопроизводительных вычислительных кластеров. Первая реализация Omni-Path с пропускной способностью 100 Гбит/с, согласно заявлениям Intel, обеспечивает меньший уровень задержки и более высокую практическую пропускную способность в сравнении с сетью Infiniband EDR[1]. Новая архитектура предлагается в качестве замены основанной на InfiniBand архитектуры Intel True Scale Fabric[2]. Компания Intel планирует развитие технологий высокопроизводительных вычислений на основе данной архитектуры вплоть до создания кластеров, преодолеющих экзафлопсный барьер к 2018–2020 гг[3][4].

Первые продукты на базе Omni-Path — адаптеры с пропускной способностью 100 Гбит/с, использующие разъемы QSFP28, и коммутаторы, построенные на базе 48-портовой ASIC Prairie River — были выпущены в ноябре 2015 года[5][6], массовые поставки начались в первом квартале 2016 года[7].

Архитектура Omni-Path появилась в лабораториях корпорации Intel после целого ряда приобретений компаний, связанных с высокопроизводительными коммуникациями: NetEffect (разработчик Ethernet-адаптеров с iWARP (англ.)русск., 2008 год)[8], Fulcrum Microsystems (чипы FocalPoint для Ethernet-коммутаторов, 2011 год)[9], активы, связанные с технологией InfiniBand компании QLogic (линейка TruScale, 2012 год)[10][11].

Omni-Path имеет обратную совместимость с инфраструктурой (программными интерфейсами) Intel TrueScale, базирующейся на технологиях InfiniBand.[12]
Omni-Path, работая на скорости 100 Гбит/c, имеет на 56 \% более низкие задержки, нежели аналогичная реализация InfiniBand. Она использует 48-портовый чип коммутатора, что позволит увеличить плотность по сравнению с 36-портовым чипом Mellanox[13][14][15]


    Межпроцессорный интерфейс Omni-Path 1-го поколения используется в третьем поколении процессоров Xeon Phi на базе архитектуры Intel MIC, готовящимся к выпуску под кодовым названием Knight’s Landing[17].
    Межпроцессорный интерфейс Omni-Path 2-го поколения будет использоваться в четвёртом поколении процессоров Xeon Phi на базе архитектуры Intel MIC, готовящимся к выпуску под кодовым названием Knight’s Hill[17].
    



