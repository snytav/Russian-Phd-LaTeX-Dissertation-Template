\chapter{Основные характеристики процессоров и GPU, использованных в тестовых расчетах} \label{AppendixA}

\section{Процессоры семейства Intel Xeon}
\subsection{Устаревшие процессоры семейства Intel Xeon}

\begin{center}
\begin{table}[ht]
\caption{Процессоры Intel Xeon с индексами E}
	\begin{tabular}{|c|c|c|c|c|c|c|}
		\hline
		Номер процессора & E5502 & E5504 & E5506 & E5520 & E5530 & E5540 \\ \hline
		Количество ядер  & 2 & \multicolumn{5}{c|}{4} \\ \hline
		Количество потоков & 2 & \multicolumn{2}{c|}{4} & \multicolumn{3}{c|}{8} \\ \hline
		Тактовая частота   &          &       &          &          &          &  \\
		процессора, ГГц    & 1,86  & 2  & 2,13 & 2,26 & 2,5 & 2,53 \\ \hline
		Кэш-память         & \multicolumn{3}{c|}{4 МБ «умный» кэш} & \multicolumn{3}{c|}{8 МБ «умный» кэш} \\ \hline
		Производительность &  \multicolumn{3}{c|}{}          & \multicolumn{3}{c|}{} \\   
		системной шины     & \multicolumn{3}{c|}{4,8 ГТ/сек} & \multicolumn{3}{c|}{5,86 ГТ/сек} \\ \hline
		Набор команд       &\multicolumn{6}{c|}{64-битный} \\ \hline
		Тепловыделение     & \multicolumn{6}{c|}{80 Вт} \\ \hline
		Диапазон напряжения & \multicolumn{6}{c|}{}  \\
		питания, VID        & \multicolumn{6}{c|}{0,75—1,35 В} \\ \hline
		Макс. объём памяти  & \multicolumn{6}{c|}{}  \\
		(зависит от типа памяти) & \multicolumn{6}{c|}{144 ГБ} \\ \hline
		Типы памяти & \multicolumn{2}{c|}{DDR3-800} & \multicolumn{2}{c|}{DDR3-800/1066} & \multicolumn{2}{c|}{DDR3-800/1066} \\ \hline
		Количество каналов & \multicolumn{6}{c|}{} \\
		памяти & \multicolumn{6}{c|}{3} \\ \hline
		Макс. пропускная & \multicolumn{6}{c|}{} \\ \hline
		способность памяти & \multicolumn{3}{c|}{19,2 ГБ/сек} & \multicolumn{3}{c|}{25,6 ГБ/сек }\\ \hline
		Расширение         & \multicolumn{6}{c|}{} \\ 
		физического адреса & \multicolumn{6}{c|}{40-битовое} \\ \hline
		\multicolumn{7}{|c|}{Применяемые технологии} \\ \hline
		Turbo Boost & \multicolumn{3}{c|}{Нет} & \multicolumn{3}{c|}{Да} \\ \hline
		Hyper-Threading & Нет & Нет & Нет & Да & Да & Да \\ \hline
		Виртуализация (VT-x) & \multicolumn{6}{c|}{Да} \\ \hline
		Виртуализации прямого & \multicolumn{6}{c|}{} \\
		ввода-вывода (VT-d)   & \multicolumn{6}{c|}{ Да} \\ \hline
		Trusted Execution &   \multicolumn{6}{c|}{ Нет} \\ \hline
	\end{tabular}
\end{table}
\end{center}



\subsubsection{Процессор Intel Xeon  E5540}
Данные этого подраздела в основном взяты с сайта \cite{geek_E5540}


\begin{table}[ht]
\begin{center}
\caption{Основные характеристики процессора Intel Xeon  E5540}
\begin{tabular}{|c|c|}
\hline	
Тактовая частота & 2.53GHz   \\ \hline
Количество ядер & 4 	     \\ \hline
кэш команд, L1 &  64 KB      \\ \hline
кэш данных, L1 &  64 KB       \\ \hline
кэш L2         &  512 KB      \\ \hline
кэш L3         &  0 KB        \\ \hline
\end{tabular}
\end{center} 	
\end{table} 	

\begin{table}[ht]
\begin{center}
\caption{Производительность процессора Intel Xeon  E5540 на операциях с плавающей точкой}
\begin{tabular}{|c|c|}
\hline	
Вычислительный алгоритм &  Производительность, GFLOPS \\ 
                 & в режиме многоядерных скалярных вычислений \\ \hline
Алгоритм Мандельброта  &  1.54 	\\ \hline
Скалярное произведение &  5.44   \\ \hline
LU-разложение          &  2.38   \\ \hline
Проверка простоты      & 0.568  \\ \hline 


		\end{tabular}
	\end{center} 	
\end{table} 	

\begin{table}[ht]
	\begin{center}
		\caption{Производительность памяти процессора Intel Xeon  E5540}
		\begin{tabular}{|c|c|}
			\hline	
			Операция  &  Производительность, GB/sec \\ \hline
			Последовательное чтение &  6.5 	\\  \hline
			Последовательная запись &  6.16   \\  \hline
			Выделение памяти средствами stdlib &  15.9 операций malloc в секунду  \\  \hline
			Запись средствами stdlib  & 7.64  \\ \hline
			Чтение средствами stdlib  & 7.53  \\ \hline 
		\end{tabular}
	\end{center} 	
\end{table} 	


%http://browser.geekbench.com/geekbench2/413456
\subsubsection{Intel Xeon 5150}

Данные этого подраздела в основном взяты с сайта \cite{geek_5150}


\begin{table}[ht]
	\begin{center}
		\caption{Основные характеристики процессора Intel Xeon  5150}
		\begin{tabular}{|c|c|}
			\hline	
			Тактовая частота & 2.72GHz   \\ \hline
			Количество ядер & 2 	     \\ \hline
			кэш команд, L1 &  0 KB      \\ \hline
			кэш данных, L1 &  0 KB       \\ \hline
			кэш L2         &  0 KB      \\ \hline
			кэш L3         &  0 KB        \\ \hline
		\end{tabular}
	\end{center} 	
\end{table} 	

\begin{table}[ht]
	\begin{center}
		\caption{Производительность процессора Intel Xeon  5150 на операциях с плавающей точкой}
		\begin{tabular}{|c|c|}
			\hline	
			Вычислительный алгоритм &  Производительность, GFLOPS \\ 
			& в режиме многоядерных скалярных вычислений \\ \hline
			Алгоритм Мандельброта  &  1.34 	\\ \hline
			Скалярное произведение &  2.12   \\ \hline
			LU-разложение          &  2.05   \\ \hline
			Проверка простоты      &  464  \\ \hline 
			
			
		\end{tabular}
	\end{center} 	
\end{table} 	

\begin{table}[ht]
	\begin{center}
		\caption{Производительность памяти процессора Intel Xeon 5150}
		\begin{tabular}{|c|c|}
			\hline	
			Операция  &  Производительность, GB/sec \\ \hline
			Последовательное чтение &  2.93 	\\  \hline
			Последовательная запись &  1.9   \\  \hline
			Выделение памяти средствами stdlib &  6.78 операций malloc в секунду  \\  \hline
			Запись средствами stdlib  & 2.03  \\ \hline
			Чтение средствами stdlib  & 1.96  \\ \hline 
		\end{tabular}
	\end{center} 	
\end{table} 	


\subsubsection{Процессор Intel Xeon E5472}
\begin{table}[ht]
	\begin{center}
		\caption{Основные характеристики процессора Intel Xeon  E5472}
		\begin{tabular}{|c|c|}
			\hline	
			Тактовая частота & 3.00GHz   \\ \hline
			Количество ядер & 4 	     \\ \hline
			кэш команд, L1 &  0 KB      \\ \hline
			кэш данных, L1 &  0 KB       \\ \hline
			кэш L2         &  6144 KB      \\ \hline
			кэш L3         &  0 KB        \\ \hline
		\end{tabular}
	\end{center} 	
\end{table} 	

\begin{table}[ht]
	\begin{center}
		\caption{Производительность процессора Intel Xeon  E5472 на операциях с плавающей точкой}
		\begin{tabular}{|c|c|}
			\hline	
			Вычислительный алгоритм &  Производительность, GFLOPS \\ 
			& в режиме многоядерных скалярных вычислений \\ \hline
			Алгоритм Мандельброта  &  6.203 	\\ \hline
			Скалярное произведение &  3.05   \\ \hline
			LU-разложение          &  2.7   \\ \hline
			Проверка простоты      &  13.9  \\ \hline 
			
			
		\end{tabular}
	\end{center} 	
\end{table} 	

\begin{table}[ht]
	\begin{center}
		\caption{Производительность памяти процессора Intel Xeon E5472}
		\begin{tabular}{|c|c|}
			\hline	
			Операция  &  Производительность, GB/sec \\ \hline
			Последовательное чтение &  2.7 	\\  \hline
			Последовательная запись &  3.4   \\  \hline
			Выделение памяти средствами stdlib &  6.78 операций malloc в секунду  \\  \hline
			Запись средствами stdlib  & 1.6  \\ \hline
			Чтение средствами stdlib  & 1.7  \\ \hline 
		\end{tabular}
	\end{center} 	
\end{table} 	



\subsubsection{Процессор Intel Xeon 5355}

\begin{table}[ht]
	\begin{center}
		\caption{Основные характеристики процессора Intel Xeon  5355}
		\begin{tabular}{|c|c|}
			\hline	
			Тактовая частота & 3.00GHz   \\ \hline
			Количество ядер & 4 	     \\ \hline
			кэш команд, L1 &  0 KB      \\ \hline
			кэш данных, L1 &  0 KB       \\ \hline
			кэш L2         &  4096 KB      \\ \hline
			кэш L3         &  0 KB        \\ \hline
		\end{tabular}
	\end{center} 	
\end{table} 	

\begin{table}[ht]
	\begin{center}
		\caption{Производительность процессора Intel Xeon  5355 на операциях с плавающей точкой}
		\begin{tabular}{|c|c|}
			\hline	
			Вычислительный алгоритм &  Производительность, GFLOPS \\ 
			& в режиме многоядерных скалярных вычислений \\ \hline
			Алгоритм Мандельброта  &  5.2 	\\ \hline
			Скалярное произведение &  2.6   \\ \hline
			LU-разложение          &  2.08   \\ \hline
			Проверка простоты      &  6.4  \\ \hline 
			
			
		\end{tabular}
	\end{center} 	
\end{table} 	

\begin{table}[ht]
	\begin{center}
		\caption{Производительность памяти процессора Intel Xeon 5355}
		\begin{tabular}{|c|c|}
			\hline	
			Операция  &  Производительность, GB/sec \\ \hline
			Последовательное чтение &  2.1 	\\  \hline
			Последовательная запись &  2.5   \\  \hline
			Выделение памяти средствами stdlib &  6.78 операций malloc в секунду  \\  \hline
			Запись средствами stdlib  & 1.36  \\ \hline
			Чтение средствами stdlib  & 1.4 \\ \hline 
		\end{tabular}
	\end{center} 	
\end{table} 	


\subsubsection{Intel Xeon 5570}

\begin{table}[ht]
	\begin{center}
		\caption{Основные характеристики процессора Intel Xeon  5570}
		\begin{tabular}{|c|c|}
			\hline	
			Тактовая частота & 2.93GHz   \\ \hline
			Количество ядер & 4 	     \\ \hline
			кэш команд, L1 &  0 KB      \\ \hline
			кэш данных, L1 &  0 KB       \\ \hline
			кэш L2         &  256 KB      \\ \hline
			кэш L3         &  0 KB        \\ \hline
		\end{tabular}
	\end{center} 	
\end{table} 	

\begin{table}[ht]
	\begin{center}
		\caption{Производительность процессора Intel Xeon  5570 на операциях с плавающей точкой}
		\begin{tabular}{|c|c|}
			\hline	
			Вычислительный алгоритм &  Производительность, GFLOPS \\ 
			& в режиме многоядерных скалярных вычислений \\ \hline
			Алгоритм Мандельброта  &  17.9 	\\ \hline
			Скалярное произведение &  29.2   \\ \hline
			LU-разложение          &  3.8  \\ \hline
			Проверка простоты      &  23.9  \\ \hline 
			
			
		\end{tabular}
	\end{center} 	
\end{table} 	

\begin{table}[ht]
	\begin{center}
		\caption{Производительность памяти процессора Intel Xeon 5570}
		\begin{tabular}{|c|c|}
			\hline	
			Операция  &  Производительность, GB/sec \\ \hline
			Последовательное чтение &  5.5 	\\  \hline
			Последовательная запись &  5.2   \\  \hline
			Выделение памяти средствами stdlib &  -  \\  \hline
			Запись средствами stdlib  & 3.9  \\ \hline
			Чтение средствами stdlib  & 8.1  \\ \hline 
		\end{tabular}
	\end{center} 	
\end{table} 	


\subsection{Современные процессоры семейства Intel Xeon}
\subsubsection{Intel Xeon 5570}
\subsubsection{Intel Xeon E5-2697}
\subsubsection{Intel XeonE5-2697Av4}

\section{Процессор Phenom II X6 1055T}
https://www.techpowerup.com/forums/threads/processor-gflops-compilation.94721/
\clearpage
\section{Графические ускорители (GPU)}



\chapter{Краткое описание основных свойств сетевого оборудования и сетевых протоколов, использованных в тестовых расчетах} \label{AppendixB}

\section{Gigabit Ethernet}
\section{Infiniband}
\section{Omnipath}


