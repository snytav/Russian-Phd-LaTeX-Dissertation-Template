\chapter{Основные характеристики процессоров и GPU, использованных в тестовых расчетах} \label{AppendixA}

\section{Процессоры семейства Intel Xeon}
\subsection{Устаревшие процессоры семейства Intel Xeon}

\begin{center}
\begin{table}[ht]
\caption{Процессоры Intel Xeon с индексами E}
\begin{tabular}{|c|c|c|c|c|c|c|}
	\hline
Название  &	Intel Xeon Processor E5502 & 	Intel Xeon Processor E5504 & 	Intel Xeon Processor E5506 & 	Intel Xeon Processor E5520 & 	Intel Xeon Processor E5530 & 	Intel Xeon Processor E5540 \\ \hline
Статус &	Производится &	Производится &	Производится &	Производится &	Производится &	Производится \\ \hline
\multicolumn{6}{|c|}{	I кв. 2009}
Дата начала серийного производства &\multicolumn{6}{|c|}{	I кв. 2009} \\ \hline
% Номер процессора 	E5502 	E5504 	E5506 	E5520 	E5530 	E5540
% Количество ядер 	2 	4
% Количество потоков 	2 	4 	8
% Базовая тактовая частота процессора 	1,86 ГГц 	2 ГГц 	2,13 ГГц 	2,26 ГГц 	2,4 ГГц 	2,53 ГГц
% Кэш-память 	4 МБ «умный» кэш 	8 МБ «умный» кэш
% Тип шины 	QPI
% Производительность системной шины 	4,8 ГТ/сек 	5,86 ГТ/сек
% Количество связей QPI 	2
% Набор команд 	64-битный
% Для встроенного применения? 	Нет 	Да 	Нет 	Нет 	Нет 	Да
% Дополнительный SKU 	Нет
% Макс. тепловыделение 	80 Вт
% Диапазон напряжения питания, VID 	0,75—1,35 В
% Цена (партия — 1000 шт.) 	$188 	$224 	$266 	$373 	$530 	$744
% Спецификация памяти
% Макс. объём памяти (зависит от типа памяти) 	144 ГБ
% Типы памяти 	DDR3-800 	DDR3-800/1066 	DDR3-800/1066
% Количество каналов памяти 	3
% Макс. пропускная способность памяти 	19,2 ГБ/сек 	25,6 ГБ/сек
% Расширение физического адреса 	40-битовое
% Поддержка памятью функции ECC 	Да 	Да 	Да 	Да 	Да 	Да
% Спецификация корпуса
% Макс. процессоров в конфигурации 	2
% Температура корпуса 		76 °C
% Размер корпуса 	42,5×45 мм
% Норма литографии техпроцесса 	45 нм
% Размер ядра процессора 	263 мм²
% Количество транзисторов в ядре 	731 млн
% Поддержка процессорного разъёма 	FCLGA1366
% Безгалогенная продукция? 	Да
% Применяемые технологии
% Turbo Boost 	Нет 	Да
% Hyper-Threading 	Нет 	Нет 	Нет 	Да 	Да 	Да
% Виртуализация (VT-x) 	Да
% Виртуализации прямого ввода-вывода (VT-d) 	Да
% Trusted Execution 	Нет
% Новые AES инструкции 	
% Intel 64 	Да
% Idle States 	Да
% Улучшенная технология SpeedStep 	Да
% Demand Based Switching 	Да
% Execute Disable Bit 	Да
\end{tabular}
\end{table}
\end{center}

\subsubsection{Процессор Intel Xeon  E5540}
Данные этого подраздела в основном взяты с сайта \cite{geek_E5540}


\begin{table}[ht]
\begin{center}
\caption{Основные характеристики процессора Intel Xeon  E5540}
\begin{tabular}{|c|c|}
\hline	
Тактовая частота & 2.53GHz   \\
Количество ядер & 4 	     \\
кэш команд, L1 &  64 KB      \\
кэш данных, L1 &  64 KB       \\
кэш L2         &  512 KB      \\
кэш L3         &  0 KB        \\
\hline                      
\end{tabular}
\end{center} 	
\end{table} 	

\begin{table}[ht]
\begin{center}
\caption{Производительность процессора Intel Xeon  E5540 на операциях с плавающей точкой}
\begin{tabular}{|c|c|}
\hline	
Вычислительный алгоритм &  Производительность, GFLOPS \\ 
                 & в режиме многоядерных скалярных вычислений \\ \hline
Алгоритм Мандельброта  &  1.54 	\\
Скалярное произведение &  5.44   \\
LU-разложение          &  64 KB   \\
Проверка простоты      & 0.568  \\ \hline

\hline
		\end{tabular}
	\end{center} 	
\end{table} 	


%http://browser.geekbench.com/geekbench2/413456
\subsubsection{Intel Xeon 5150}
\subsubsection{Intel Xeon E5472}
\subsubsection{Intel Xeon 5355}
\subsubsection{Intel Xeon 5570}
\subsubsection{Intel Xeon E5472}


\subsection{Современные процессоры семейства Intel Xeon}
\subsubsection{Intel Xeon 5570}
\subsubsection{Intel Xeon E5-2697}
\subsubsection{Intel XeonE5-2697Av4}

\section{Процессор Phenom II X6 1055T}
https://www.techpowerup.com/forums/threads/processor-gflops-compilation.94721/

\section{Графические ускорители (GPU)}



\chapter{Краткое описание основных свойств сетевого оборудования и сетевых протоколов, использованных в тестовых расчетах} \label{AppendixB}

\section{Gigabit Ethernet}
\section{Infiniband}
\section{Omnipath}


